\section{Introduction}\label{section-introduction}
Intel System on a Chip (\gls{soc}) features a new set of Intel Uncore Intellectual Property (IP) for every generation.
Section \ref{section-introduction} covers the introduction and overview of BIOS, UEFI and it's role and major components - Advanced Configuration and Power Interface (\gls{acpi}), Peripheral Component Interconnect Express (\gls{pcie}) and Graphics Controller. Section \ref{section-design} describes the design of UEFI and the boot phases in detail. The study of the BIOS binary structure and mapping of each components byte and alignment is described in Section \ref{section-architecture}. Proposed work to reducing the process of build iteration described in Section \ref{section-proposed-work}. 

\subsection{Uncore Intellectual Properties}
The Uncore encompasses system agent (SA), memory and Uncore agents such as graphics controller, display controller, memory controller and Input Output (IO). The Uncore IPs are Peripheral Component Interface Express (PCIe), Graphics Processing Engine (GPE), Thunderbolt, Imaging Processing Agent (IPU), North Peak (NPK), Virtualization Technology for directed-IO (Vt-d), Volume Management Device (VMD).

PCI Express abbreviated as \gls{pci} or \gls{pcie}, is designed to replace the older PCI standards.
A data communication system is developed for use the transfer data between the host and the
peripheral devices via PCIe. Thunderbolt is the brand name of a hardware interface developed
by Intel that allows the connection of external peripherals to a computer. Thunderbolt combines
PCI Express (PCIe) and DisplayPort (DP) into two serial signals, and additionally provides DC
power, all in one cable. Graphics Processing Engine (GPE), Integrated graphics, shared graphics
solutions, integrated graphics processors (IGP) or unified memory architecture (UMA) utilize a
portion of a computer's system RAM rather than dedicated graphics memory. GPEs can be
integrated onto the motherboard as part of the chipset. Virtual Technology for Directed-IO (Vt-d)
is an input/output memory management unit (IOMMU) allows guest virtual machines to directly
use peripheral devices, such as Ethernet, accelerated graphics cards, and hard-drive controllers,
through DMA and interrupt remapping.

\subsection{Legacy \gls{bios} and \gls{uefi}}

\paragraph{\gls{bios}} is the dominant standard which defines a firmware interface.

"Legacy" (as in Legacy \gls{bios}), in the context of firmware specifications, refer to an older, widely used specification. Major responsibility of \gls{bios} is to set up the hardware, load and start an \gls{os}. When the system boots, the BIOS initializes and identifies system devices including video display card, mouse, hard disk drive, keyboard, solid state drive and other hardware followed by locating software held on a boot device i.e. a hard disk or removable storage such as CD/DVD or USB and loads and executes that software, giving it control of the computer. This process is also referred to as "booting" or "boot strapping".

\subsubsection{Background of Legacy \gls{bios}}
In 1980s, IBM developed the personal computer with a 16-bit BIOS with the aim of ending the BIOS after the first 250,000 products. Legacy BIOS is based upon Intel's original 16-bit architecture, ordinarily referred to as  "8086" architecture. And as technology advanced, Intel extended that 8086 architecture from 16 to 32-bit.
Legacy BIOS is able to run different \gls{os}, such as MS-DOS, equally well on systems other than IBM. Additionally, Legacy BIOS has a defined OS-independent interface for hardware that enables interrupts to communicate with video, disk and keyboard services along with the BIOS ROM loader and bootstrap loader, to name a few.

Use of legacy BIOS is diminishing and is expected to be phased out in new systems by the year 2020.

\subsubsection{Limitations of legacy BIOS}
Over the years, many new configuration and power management technologies were integrated
into BIOS implementations as well as support for many generations of Intel® architecture
hardware. However certain limitations of BIOS implementations such as 16-bit addressing mode,
1 MB addressable space, PC AT hardware dependencies and upper memory block (UMB)
dependencies persisted throughout the years. The industry also began to have need for methods to
ensure quality of individual firmware modules as well as the ability to quickly integrate libraries
of third-party firmware modules into a single platform solution across multiple product lines.
These inherent limitations and existing market demands opened the opportunity for a fresh BIOS
architecture to be developed and introduced to the market. The UEFI specifications and resulting
implementations have begun to effectively address these persisting market needs.

One of the critical maintenance challenges for BIOS is that each implementation has tended to
be highly customized for the specific motherboard on which it is deployed. Moving component
modules across designs typically requires significant porting, integration, testing and debug work.
This is one of the markets challenges the UEFI architecture promises to address.

\subsection{Unified Extensible Firmware Interface (\gls{uefi})}
\gls{uefi} was developed as a replacement for legacy BIOS to streamline the booting process, and act as the interface between a operating system and its platform firmware. It not only replaces most BIOS functions, but also offers a rich extensible pre-OS environment with advanced boot and runtime services.
Unified Extensible Firmware Interface (\gls{uefi}) is grounded in Intel's initial Extensible Firmware Interface (EFI) specification 1.10, which defines a software interface between an operating system and platform firmware. The UEFI architecture allows users to execute applications on a command line interface. It has intrinsic networking capabilities and is designed to work with multi-processors (MP) systems.

\begin{figure}[h]
	\includegraphics[width=\linewidth]{uefi_board_of_directors}
	\caption{Board of Directors of UEFI Forum}\label{fig:introduction-uefi-board-of-directors}
\end{figure}

The UEFI Forum board of directors consists of representatives from 11 industry leaders as described in Figure \ref{fig:introduction-uefi-board-of-directors}. These involved organizations work to ensure that the UEFI specifications meet industry needs.

UEFI uses a different interface for boot services and runtime services but UEFI does not specify how "Power On Self Test" (POST) and Setup are implemented - those are BIOS' primary functions.

\subsubsection{\gls{uefi} Driver Model Extension}
Access to boot devices is provided through a set of protocol interfaces. One purpose of the
UEFI Driver Model is to provide a replacement for \verb|PC-AT|-style option ROMs. It is important
to point out that drivers written to the UEFI Driver Model are designed to access boot devices in
the pre-boot environment. They are not designed to replace the high-performance, OS-specific
drivers.

The UEFI Driver Model is designed to support the execution of modular pieces of code,
also known as drivers, that run in the pre-boot environment. These drivers may manage or control
hardware buses and devices on the platform, or they may provide some software-derived, platform specific service. The UEFI Driver Model also contains information required by UEFI driver writers to design and implement any combination of bus drivers and device drivers that a platform
might need to boot a UEFI-compliant OS.

The UEFI Driver Model is designed to be generic and can be adapted to any type of bus or
device. The UEFI Specification describes how to write PCI bus drivers, PCI device drivers, USB
bus drivers, USB device drivers, and SCSI drivers. Additional details are provided that allow UEFI
drivers to be stored in PCI option ROMs, while maintaining compatibility with legacy option ROM
images.

One of the design goals in the UEFI Specification is keeping the driver images as small as
possible. However, if a driver is required to support multiple processor architectures, a driver
object file would also be required to be shipped for each supported processor architecture. To
address this space issue, this specification also defines the EFI Byte Code Virtual Machine. A
UEFI driver can be compiled into a single EFI Byte Code object file. UEFI Specificationcomplaint firmware must contain an EFI Byte Code interpreter. This allows a single EFI Byte
Code object file that supports multiple processor architectures to be shipped. Another space saving
technique is the use of compression. This specification defines compression and decompression
algorithms that may be used to reduce the size of UEFI Drivers, and thus reduce the overhead
when UEFI Drivers are stored in ROM devices.

The information contained in the UEFI Specification can be used by OSVs, IHVs, OEMs,
and firmware vendors to design and implement firmware conforming to this specification, drivers
that produce standard protocol interfaces, and operating system loaders that can be used to boot
UEFI compliant operating systems.

\subsubsection{\gls{uefi}'s Role in boot process}

During the boot process, UEFI speaks to the operating system loader and acts as the interface between the operating system and the BIOS.

The \verb|PC-AT| boot environment presents significant challenges to innovation within the
industry. Each new platform capability or hardware innovation requires firmware developers to
craft increasingly complex solutions, and often requires OS developers to make changes to their
boot code before customers can benefit from the innovation. This can be a time-consuming process
requiring a significant investment of resources. The primary goal of the UEFI specification is to
define an alternative boot environment that can alleviate some of these considerations. In this goal, the specification is like other existing boot specifications.

\subsection{Comparing of Legacy \gls{bios} and \gls{uefi}}

\begin{table}
	\centering
	\renewcommand{\arraystretch}{2}
	\caption{Legacy BIOS v/s UEFI}\label{table:legacy-bios-vs-uefi}
	\begin{tabular}{l | p{5cm} | p{5cm}}
		& Legacy BIOS & EFI
		\\ \hline \hline
		Language & Assembly & C ($ 99\% $)
		\\ \hline
		Resource & Interrupt Hardcode Memory Access hardcore I/O Access & Diver, Protocols
		\\ \hline
		Processor & x86 16-bit & CPU Protects Mode (Flat Mode)
		\\ \hline
		Expand & Hook Interrupt & Load Driver
		\\ \hline
		OS Bridge & ACPI & Run Time Driver Software
		\\ \hline
		$ 3^{rd} $ Party ISV \& IHV & Bas for Support & Easy for Support and for Multi Platforms
		\\ \hline
	\end{tabular}
\end{table}


\subsection{Advanced Configuration and Power Interface (\gls{acpi})}
The \say{ACPI Component Architecture (\gls{acpica})} is an implementation of a group of software components according to the ACPI specification. It is created with the goal of isolating operating system dependencies to a relatively small translation or conversion layer (the OS Services Layer). This makes the bulk of the ACPICA code independent of any individual operating system.so it can used for new operating systems with no source changes within the ACPICA code itself.

Tthe architecture include below component:
\begin{itemize}
  \item \gls{acpica} Subsystem - independent of OS and kernel which serves the primal ACPI services like the AML interpreter and management of namespace.
  \item \gls{acpica} Subsystem - independent of OS and OS Services Layer for every host OS to serve OS support.
  \item The ASL compiler/disassembler for translating the source code from ASL to AML and also disassembling the ASL source code from the binary ACPI tables if exists.
  \item Many ACPI utilities for running the interpreter in level 3 user space taking out the binary ACPI tables residing in the output result of ACPI Dump utility along with translating ACPICA source code to output format of Linux/Unix.
\end{itemize}

Figure \ref{fig:introduction-acpi-component-architecture} portrays the ACPICA subsystem in relation with the device driver(s), host OS, and the ACPI hardware.

\begin{figure}[!htbp]
	\centering
	\includegraphics[width=0.8\linewidth]{introduction/acpi-component-architecture}
	\caption{The \gls{acpi} Component Architecture}\label{fig:introduction-acpi-component-architecture}
\end{figure}

\subsubsection{Overview of \gls{acpica} Subsystem}
The \say{\gls{acpica} Subsystem} develops the basic primal aspects of the ACPI specification. Includes an AML parser/interpreter, ACPI table and device support, ACPI namespace management, and event handling. As the ACPICA subsystem serves the lower level services for system, it also involves low-level services of OS like memory management, scheduling, synchronization and I/O.

To allow the ACPICA Subsystem to easily link between any operating system that engage such services, an Operating System Services Layer transforms ACPICA-to-OS requests inside the system calls publicized by the host OS. This OS Services Layer is the one and only element of the ACPICA which pertains source code which is limited to a particular host OS.

%\subsubsection{OS-independent ACPICA Subsystem}
%The OS-independent ACPICA Subsystem supplies the major building blocks or subcomponents that are required for all ACPI implementations — including an AML interpreter, a namespace manager, ACPI event and resource management, and ACPI hardware support.
%
%One of the goals of the ACPICA Subsystem is to provide an abstraction level high enough such
%that the host operating system does not need to understand or know about the very low-level ACPI
%details. For example, all AML code is hidden from the host. Also, the details of the ACPI hardware
%are abstracted to higher-level software interfaces.
%
%The ACPICA Subsystem implementation makes no assumptions about the host operating system or environment. The only way it can request operating system services is via interfaces provided by the OS Services Layer.
%
%The primary user of the services provided by the ACPICA Subsystem are the host OS device drivers and power/thermal management software.
%
\subsubsection{Operating System Services Layer \gls{osl}}
\say{OS Services Layer (OSL)} act as a request translation service for host os from OS-independent ACPICA subsystem. The OSL develops a common subset for interfaces of OS service by utilizing the primitives usable from host OS.

The OSL has to be developed afresh for each and every supported host OS. There exists only one ACPICA Subsystem which OS-independent but there has to be a different OSL for each OS backed by the ACPICA.

The whole ACPICA in relation to the host OS is portrayed in Figure \ref{fig:introduction-acpica-subsystem-architecture}

\begin{figure}[!htbp]
	\centering
	\includegraphics[width=0.7\linewidth]{introduction/acpica-subsystem-architecture}
	\caption{ACPICA Subsystem Architecture}\label{fig:introduction-acpica-subsystem-architecture}
\end{figure}

\subsubsection{\gls{acpica} Subsystem Interaction}
ACPICA Subsystem develops a subset of external interface links that could directly summoned via host OS. These Acpi interfaces serve the literal ACPI services for host. When OS services are needed while servicing of request of an ACPI the Subsystem makes oblique request to host OS through the fixed AcpiOs interfaces. 

\begin{figure}[!htbp]
  \centering
  \includegraphics[width=0.7\linewidth]{introduction/acpi-interaction-between-the-architectural-components}
  \caption{Interaction between the Architectural Components}\label{fig:-introduction-acpi-interaction-between-the-architectural-components}
\end{figure}

Figure \ref{fig:-introduction-acpi-interaction-between-the-architectural-components} portrays the kinship and fundamental interaction linking the diverse architectural modules by screening the control flow among them. Note that OS independent ACPICA Subsystem could never call the host OS directly and instead it has to make call(s) to the AcpiOs interfaces inside the OSL. This serves the ACPICA code as OS-independence.

\subsection{Peripheral Component Interconnect Express (\gls{pcie})}
The PCI architecture has proven to be successful beyond even the most optimistic expectations. Today nearly every new computer platform comes outfitted with multiple PCI slots. In addition to the unprecedented number of PCI slots being shipped, there are also hundreds of PCI adapter cards that are available to satisfy virtually every conceivable application. This enormous momentum is difficult to ignore.

Today there is also a need for a new, higher-performance I/O interface to support emerging, ultrahigh-bandwidth technologies such as 10 Gigabit Ethernet, 10 Gigabit FibreChannel, 4X and 12X InfiniBand, and others. A standard that can meet these performance objectives, while maintaining backward compatibility to previous generations of PCI would undoubtedly provide the ideal solution.

To meet these objectives, the PCI-X 2.0 standard has been developed. PCI-X 2.0 has the performance to feed the most bandwidth-hungry applications while at the same time maintaining complete hardware and software backward compatibility to previous generations of PCI and PCI-X. The PCI-X 2.0 standard introduces two new speed grades: PCI-X 266 and PCI-X 533. These speed grades offer bandwidths that are two times and four times that of PCI-X 133 -- ultimately providing bandwidths that are more than 32 times faster than the original version of PCI that was introduced eight years ago. It achieves the additional performance via time-proven DDR (Double Data Rate) and QDR (Quad Data Rate) techniques that transmit data at either 2-times or 4-times the base clock frequency. Because PCI-X 2.0 preserves so many elements from previous generations of PCI it is the beneficiary of a tremendous amount of prior development work. The operating systems, connector, device drivers, form factor, protocols, BIOS, electrical signaling, BFM (bus functional model), and other original PCI elements, are all heavily leveraged in the PCI-X 2.0 specification.

In fact, many of these elements remain identical in PCI-X 2.0. These similarities make implementation easy because these elements have already been designed and engineers are already familiar with them. As a result, the time-to-market is short, and risk is dramatically reduced.

The market migration to PCI-X 2.0 will also be easy because there are so many previous-generation PCI adapter cards already on the market. There are already hundreds of PCI adapter cards that are available today that can be utilized by every PCI-X 266 and PCI-X 533 slot. In addition, new PCI-X 266 and PCI-X 533 adapter cards have ready homes in any of the millions of PCI and PCI-X slots in existing systems. Because of these factors, PCI-X 2.0 provides the ideal next-generation, local I/O solution for high-bandwidth applications. It offers the performance needed for today’s and tomorrow’s applications in an easy-to-adopt, backward-compatible standard.

\subsubsection{Functional Description}
PCI BIOS functions provide a software interface to the hardware used to implement a PCI based system. Its primary usage is for generating operations in PCI specific address spaces (configuration space and Special Cycles).
PCI BIOS functions are specified to operate in the following modes of the X86 architecture. The modes are: real-mode, 16:16 protected mode (also known as 286 protected mode), 16:32 protected mode (introduced with the 386), and 0:32 protected mode (also known as “flat” mode, wherein all segments start at linear address 0 and span the entire 4-GB address space).

Access to the PCI BIOS functions for 16-bit callers is provided through Interrupt 1Ah. 32-bit (i.e., protected mode) access is provided by calling through a 32-bit protected mode entry point. The PCI BIOS function code is B1h. Specific BIOS functions are invoked using a sub-function code. A user simply sets the host processors registers for the function and sub-function desired and calls the PCI BIOS software. Status is returned using the CARRY FLAG ([CF]) and registers specific to the function invoked.

\subsubsection{UEFI PCI Services}
UEFI stands for Unified Extensible Firmware Interface. The UEFI Specification, Version 2.3 or later describes an interface between the operating system and the platform firmware. The interface is in the form of data tables that contain platform-related information and boot and run-time services calls that are available to the operating system and its loader. Together, these provide a standard environment for booting an operating system.

The following sections provide an overview of the EFI Services relevant to PCI (including Conventional PCI, PCI-X, and PCI Express). For details, refer to the UEFI Specification. UEFI is processor-agnostic.


\subsubsection{UEFI Driver Model}
The UEFI Driver Model is designed to support the execution of drivers that run in the pre-boot environment present on systems that implement the UEFI firmware. These drivers may manage and control hardware buses and devices on the platform, or they may provide some software derived platform specific services.

The UEFI Driver Model is designed to extend the UEFI Specification in a way that supports device drivers and bus drivers. It contains information required by UEFI driver writers to design and implement any combination of bus drivers and device drivers that a platform may need to boot an UEFI-compliant operating system.

Applying the UEFI Driver Model to PCI, the UEFI Specification defines the PCI Root Bridge Protocol and the PCI Driver Model and describes how to write PCI bus drivers and PCI devices drivers in the UEFI environment. For details, refer to the UEFI Specification.

\begin{itemize}
	\item \textbf{PCI Root Bridge Protocol} - A PCI Root Bridge is represented in UEFI as a device handle that contains a Device Path Protocol instance and a PCI Root Bridge Protocol instance. PCI Root Bridge Protocol provides an I/O abstraction for a PCI Root Bridge that the host bus can perform. This protocol is used by a PCI Bus Driver to perform PCI Memory, PCI I/O, and PCI Configuration cycles on a PCI Bus. It also provides services to perform different types of bus mastering DMA on a PCI bus. PCI Root Bridge Protocol abstracts device specific code from the system memory map. This allows system designers to make changes to the system memory map without impacting platform independent code that is consuming basic system resources. An example of such system memory map changes is a system that provides non-identity memory mapped I/O (MMIO) mapping between the host processor view and the PCI device view.
	\item \textbf{PCI Driver Model} - The PCI Driver Model is designed to extend the UEFI Driver Model in a way that supports PCI Bus Drivers and PCI Device Drivers. This applies to Conventional PCI, PCI-X, and PCI Express. PCI Bus Drivers manage PCI buses present in a system. The PCI Bus Driver creates child device handles that must contain a Device Path Protocol instance and a PCI I/O Protocol instance. The PCI I/O Protocol is used by the PCI Device Driver to access memory and I/O on a PCI controller. PCI Device Drivers manage PCI controllers present on PCI buses. The PCI Device Drivers produce an I/O abstraction that may be used to boot an UEFI compliant operating system.
\end{itemize}

\subsubsection{BUS Performances and Number of Slots Compared}
The various architectures defined by the PCISIG. The table shows the evolution of bus frequencies and bandwidths., as it obvious, increasing bus frequency compromises the number of electrical loads or number of connectors allowable on a bus at that frequencies. At some point for a given bus architecture there is an upper limit beyond which one cannot further increase the bus frequency, hence requiring the definition of a new bus architecture.

\begin{figure}[!htbp]
	\centering
	\includegraphics[width=\linewidth]{introduction/comparison-of-bus-frequency-bandwidth-slots}
	\caption{Comparison of Bus Frequency, Bandwidth and Number of Slots}\label{fig:comparison-of-bus-frequency-bandwidth-slots}
\end{figure}

A PCI express (PCIe) Interconnect that connects two devices is referred to as a Link. A link consists if either x1, x2, x4, x8, x12, x16 or x32 signal pairs in each direction. These signals are referred to as Lanes. A designer determines how many Lanes to implement based on the targeted performance benchmark required on a given Link.


The Figure \ref{fig:comparison-of-bus-frequency-bandwidth-slots} shows aggregate bandwidth numbers for various Link width implementations, as is apparent from this table, the peak bandwidth achievable with \gls{pcie} is significantly higher than any existing bus today.



\subsection{Graphics Controller}
Almost every graphics controllers are merely PCI controllers only. And it is also obvious that the graphics drivers who are responsible to control and manage these graphics controllers are also PCI drivers. Note that even if the most graphics controllers are PCI controllers but even then the graphics controllers can also utilize many of the other buses i.e. USB buses. 

Characterizes of Graphics drivers are listed below:
\begin{itemize}
	\item Follows UEFI Driver Modal
	\item Depending on the driver manged adapter, a graphics driver could be classified as into: a single output adapter and a multiple output adapter.
	\item For each output expected, the graphics driver has to construct child handles.
	\item For some of the output ports and protocols (such as GOP Protocol) the graphics drivers must create child handles.
	\item Graphics drivers are chip-specific because of the requirement to initialize and manage the graphics device.
\end{itemize}
Note that (\gls{ihv}) has privilege for choosing whether to support and implement all the required modules of the UEFI specification. i.e., all modules might not be implemented to support on a specified system configuration which doesn't support all of the services and features understood by the needed modules.

\subsubsection{Graphics Output Protocol (\gls{gop})}
The \say{Graphics Output Protocol \gls{gop}} Driver is member of the driver of UEFI boot time which are responsible for running up the display while the bios is booting. This driver triggers displaying of logo while the bios is booting.

\subsubsection{GOP Overview}
The GOP driver is the successor for video controller of legacy BIOS and sheers the utilization of UEFI pre-boot firmware without the use of CSM. The GOP driver can be $ 32-bit $, $ 64-bit $, or $ IA-64 $ with no binary support. Pre-boot firmware architecture of UEFI which could be either $ 32-bit $ or $ 64-bit $ has to adapt the corresponding GOP driver architecture ($ 32-bit $ or $ 64-bit $). The GOP driver could be one of the boot mode: \say{fastboot} (for specific platform optimized mode to speedup the boot time) or \say{generic} (the normal boot process).

\subsubsection{GOP DRIVER}
The EFI specification characterizes the \say{Universal Graphic Adapter (UGA)} protocol to provide graphics which could be device-independent. However, Specification of UEFI eliminated the inclusion of UGA and replaced it with it's successor \gls{gop} so that VGA hardware dependencies can be removed.

\subsubsection{GOP Integration}
The platform firmware must meet the following requirements for GOP Driver integration:
\begin{itemize}
	\item Platform firmware must be compliant to UEFI 2.1 or later.
	\item Platform must enumerate and initialize the graphics device.
	\item Platform must allocate enough graphics frame buffer memory required to support the native mode resolution of the integrated display.
	\item The platform must produce the standard \verb|EFI_PCI_IO_PROTOCOL| and as well as the \verb|EFI_DEVICE_PATH_PROTOCOL| on the graphics device handle. Additionally, the platform must produce \verb|PLATFORM_GOP_POLICY_PROTOCOL|.
	\item The platform firmware must not launch the legacy Video BIOS.
\end{itemize}

The GOP Driver solution comprises the following files shown in Table \ref{table:gop-driver-files} GOP driver files.

\begin{table}
	\centering
	\renewcommand\arraystretch{2}
	\caption{\gls{gop} Driver files}\label{table:gop-driver-files}
	\begin{tabular}{l | p{5cm} | p{5cm}}
		File Name & Description & Format
		\\ \hline \hline
		\verb|GopDriver.efi| & The \gls{gop} driver binary & Uncompressed PE/COFF image
		\\ \hline
		\verb|Vbt.bin| & Contains Video BIOS Table (VBT) data & Raw Binary
		\\ \hline
		\verb|Vbt.bsf| & BMP script file. Required for modifying Vbt.bin using BMP tool & Text
		\\ \hline
	\end{tabular}
\end{table}

Customize the VBT data file\verb| Vbt.bin| as per platform requirements and the corresponding BSF file. Integrate \verb|Vbt.bin| and \verb|GopDriver.efi| files into the platform firmware image. The process of accomplishing this step is determined by the platform implementer, specific to the platform firmware implementation.

