%%%%%%%%%%%%%%%%%%%%%%%%%%%%%%%%%%%%%%%%%
% Journal Article
% 
% Gahan M. Saraiya
% 18MCEC10
%
% References
% ==========
% 
%%%%%%%%%%%%%%%%%%%%%%%%%%%%%%%%%%%%%%%%%
\documentclass[a4paper,11pt,oneside]{Thesis}
%\usepackage[a4paper]{geometry}
\usepackage[utf8]{inputenc}
\usepackage[english]{babel}
%\usepackage{lipsum}
%\usepackage{listings}

\usepackage{graphicx}
\graphicspath{ {./assets/} {./pics/} {./eps/} {./figures/}}
%\usepackage{epsfig}
%\usepackage{epstopdf}



%%%%%%%%%%%%%%%%%%%%%%%%%%%%%%%%%%%%%%%%%%%%%%%%%%
%% COLOR DEFINITIONS
%%%%%%%%%%%%%%%%%%%%%%%%%%%%%%%%%%%%%%%%%%%%%%%%%%
\usepackage[svgnames]{xcolor} % Enabling mixing colors and color's call by 'svgnames'
%%%%%%%%%%%%%%%%%%%%%%%%%%%%%%%%%%%%%%%%%%%%%%%%%%
\definecolor{MyColor1}{rgb}{0.2,0.4,0.6} %mix personal color
\newcommand{\textb}{\color{Black} \usefont{OT1}{lmss}{m}{n}}
\newcommand{\blue}{\color{MyColor1} \usefont{OT1}{lmss}{m}{n}}
\newcommand{\blueb}{\color{MyColor1} \usefont{OT1}{lmss}{b}{n}}
\newcommand{\red}{\color{LightCoral} \usefont{OT1}{lmss}{m}{n}}
\newcommand{\green}{\color{Turquoise} \usefont{OT1}{lmss}{m}{n}}
%%%%%%%%%%%%%%%%%%%%%%%%%%%%%%%%%%%%%%%%%%%%%%%%%%




%%%%%%%%%%%%%%%%%%%%%%%%%%%%%%%%%%%%%%%%%%%%%%%%%%
%% FONTS AND COLORS
%%%%%%%%%%%%%%%%%%%%%%%%%%%%%%%%%%%%%%%%%%%%%%%%%%
%    SECTIONS
%%%%%%%%%%%%%%%%%%%%%%%%%%%%%%%%%%%%%%%%%%%%%%%%%%
%\usepackage{titlesec}
%\usepackage{sectsty}
%%%%%%%%%%%%%%%%%%%%%%%%
%%set section/subsections HEADINGS font and color
%\sectionfont{\color{MyColor1}}  % sets colour of sections
%\subsectionfont{\color{MyColor1}}  % sets colour of sections
%
%%set section enumerator to arabic number (see footnotes markings alternatives)
%\renewcommand\thesection{\arabic{section}.} %define sections numbering
%\renewcommand\thesubsection{\thesection\arabic{subsection}} %subsec.num.
%
%%define new section style
%\newcommand{\mysection}{
%    \titleformat{\section} [runin] {\usefont{OT1}{lmss}{b}{n}\color{MyColor1}} 
%    {\thesection} {3pt} {} } 

% Glossaries build
\usepackage[acronym]{glossaries}
\makeglossaries
\loadglsentries{sections/glossaries}


\usepackage{longtable}
\renewcommand\thesection{\Roman{section}} % Roman numerals for the sections
\renewcommand\thesubsection{\Roman{subsection}} % Roman numerals for subsections
%----------------------------------------------------------------------------------------
%       DATE FORMAT
%----------------------------------------------------------------------------------------
\usepackage{datetime}
\newdateformat{monthyeardate}{\monthname[\THEMONTH], \THEYEAR}
%----------------------------------------------------------------------------------------

%\titleformat{\section}[block]{\large\scshape\centering}{\thesection.}{1em}{} % Change the look of the section titles
%\titleformat{\subsection}[block]{\large}{\thesubsection.}{1em}{} % Change the look of the section titles
%\newcommand{\horrule}[1]{\rule{\linewidth}{#1}} % Create horizontal rule command with 1 argument of height
%\usepackage{fancyhdr} % Headers and footers
%\pagestyle{fancy} % All pages have headers and footers
%\fancyhead{} % Blank out the default header
%\fancyfoot{} % Blank out the default footer



%%%%%%%%%%%%%%%%%%%%%%%%%%%%%%%%%%%%%%%%%%%%%%%%%%
%		CAPTIONS
%%%%%%%%%%%%%%%%%%%%%%%%%%%%%%%%%%%%%%%%%%%%%%%%%%
%\usepackage{caption}
%%\usepackage{subcaption}
%%%%%%%%%%%%%%%%%%%%%%%%%
%\captionsetup[figure]{labelfont={color=Turquoise}}

%%%%%%%%%%%%%%%%%%%%%%%%%%%%%%%%%%%%%%%%%%%%%%%%%%
%		!!!EQUATION (ARRAY) --> USING ALIGN INSTEAD
%%%%%%%%%%%%%%%%%%%%%%%%%%%%%%%%%%%%%%%%%%%%%%%%%%
%using amsmath package to redefine eq. numeration (1.1, 1.2, ...) 
%%%%%%%%%%%%%%%%%%%%%%%%

%\renewcommand{\theequation}{\thesection\arabic{equation}}
%
%%set box background to grey in align environment 
%\usepackage{etoolbox}% http://ctan.org/pkg/etoolbox
%\makeatletter
%\patchcmd{\@Aboxed}{\boxed{#1#2}}{\colorbox{black!15}{$#1#2$}}{}{}%
%\patchcmd{\@boxed}{\boxed{#1#2}}{\colorbox{black!15}{$#1#2$}}{}{}%
%\makeatother

%%%%%%%%%%%%%%%%%%%%%%%%%%%%%%%%%%%%%%%%%%%%%%%%%%

% -------------------------------------------------------------------------------
% *** FLOWCHART AND GRAPHS PACKAGES ***
% -------------------------------------------------------------------------------
\usepackage{tikz}
\usepackage{pgfplots}
\usepackage{neuralnetwork}
\pgfplotsset{compat=1.5.1}
\usetikzlibrary{snakes, arrows, shapes, shapes.geometric, calc, automata, positioning}
\tikzstyle{startstop} = [rectangle, rounded corners, minimum height=1cm, minimum width=2cm, 
text centered, trapezium stretches=true, draw=black, 
%fill=red!30
]

\tikzstyle{io} = [trapezium, trapezium left angle=70, trapezium right angle=110, minimum width=3cm, trapezium stretches=true, minimum height=1cm, text centered, draw=black, 
%fill=blue!30
]
\tikzstyle{process} = [rectangle, minimum width=3cm, minimum height=1cm, text centered, draw=black, trapezium stretches=true, 
%fill=orange!30
]
\tikzstyle{decision} = [diamond, minimum width=3cm, minimum height=1cm, text centered, draw=black, trapezium stretches=true, 
%fill=green!30
]
\tikzstyle{arrow} = [thick,->,>=stealth]
\pgfplotsset{every axis/.append style={tick label style={/pgf/number format/fixed},font=\scriptsize,ylabel near ticks,xlabel near ticks,grid=major}}
\tikzset{%
	every neuron/.style={
		circle,
		draw,
		minimum size=1cm
	},
	neuron missing/.style={
		draw=none, 
		scale=4,
		text height=0.333cm,
		execute at begin node=\color{black}$\vdots$
	},
}
% -------------------------------------------------------------------------------

% -------------------------------------------------------------------------------
% *** INDIAN RUPEE SYMBOL ***
% -------------------------------------------------------------------------------
\usepackage{tfrupee} 
% ------------ INDIAN RUPEE SYMBOL END ---------------------------


% -------------------------------------------------------------------------------
% SET UP MATH Commands and configs
% -------------------------------------------------------------------------------


\usepackage{amsmath, amssymb, amsfonts, amsthm, fouriernc, mathtools}
% mathtools for: Aboxed (put box on last equation in align envirenment)
%\usepackage{microtype} %improves the spacing between words and letters

\usepackage{theorem}
% Special Matrix
\newenvironment{spmatrix}[1]
{\def\mysubscript{#1}\mathop\bgroup\begin{bmatrix}}
	{\end{bmatrix}\egroup_{\textstyle\mathstrut\mysubscript}}
% Adding explaination below equation terms
\newcommand{\explain}[2]{\underbrace{#1}_{\parbox{\widthof{$#1$}}{\tiny#2}}}
%\newcommand{\explain}[2]{\underbrace{#1}_{\parbox{\widthof{#1}}{\footnotesize\raggedright #2}}}

%%%%%%%%%%%%%%%%%%%%%%%%%%%%%%%%%%%%%%%%%%%%%%%%%%
%% DESIGN CIRCUITS
%%%%%%%%%%%%%%%%%%%%%%%%%%%%%%%%%%%%%%%%%%%%%%%%%%
%\usepackage[siunitx, american, smartlabels, cute inductors, europeanvoltages]{circuitikz}
%%%%%%%%%%%%%%%%%%%%%%%%%%%%%%%%%%%%%%%%%%%%%%%%%%



%\makeatletter
%\let\reftagform@=\tagform@
%\def\tagform@#1{\maketag@@@{(\ignorespaces\textcolor{red}{#1}\unskip\@@italiccorr)}}
%\renewcommand{\eqref}[1]{\textup{\reftagform@{\ref{#1}}}}
%\makeatother
%\usepackage{hyperref}
%\hypersetup{colorlinks=true}

% to allow adding line break in table cell

\usepackage{makecell}
%\hypersetup{%
%  colorlinks=true,
%  linkcolor=blue,
%  linkbordercolor={0 0 1}
%}
% 
\renewcommand\lstlistingname{Algorithm}
\renewcommand\lstlistlistingname{Algorithms}
\def\lstlistingautorefname{Alg.}

\lstdefinestyle{Python}{
    language        = Python,
    frame           = lines, 
    basicstyle      = \footnotesize,
    keywordstyle    = \color{blue},
    stringstyle     = \color{green},
    commentstyle    = \color{red}\ttfamily
}

%\setlength{\parindent}{0.0in}
%\setlength{\parskip}{0.05in}

%%%%%%%%%%%%%%%%%%%%%%%%%%%%%%%%%%%%%%%%%%%%%%%%%%
%% PREPARE TITLE
%%%%%%%%%%%%%%%%%%%%%%%%%%%%%%%%%%%%%%%%%%%%%%%%%%
\title{
%	\blue Project Report \\
    \blueb Generic IP independent BIOS Signing and Parsing}
\author{Gahan Saraiya (18MCEC10)}
\date{\monthyeardate\today}
%\date{March, 2019}

%%%%%%%%%%%%%%%%%%%%%%%%%%%%%%%%%%%%%%%%%%%%%%%%%%
%----------------------------------------------------------------------------------------
%       SET HEADER AND FOOTER
%----------------------------------------------------------------------------------------
\newcommand\theauthor{Gahan Saraiya}
\newcommand\thesubject{Generic IP independent BIOS Signing and Parsing}
\renewcommand{\footrulewidth}{0.4pt}% default is 0pt
\fancyhead[C]{Institute of Technology, Nirma University $\bullet$ \monthyeardate\today} % Custom header text
\fancyfoot[LE,LO]{\thesubject}
\fancyfoot[RO,LE]{Page \thepage} % Custom footer text
%----------------------------------------------------------------------------------------

\renewcommand\theadalign{bc}
\renewcommand\theadfont{\bfseries}
\renewcommand\theadgape{\Gape[4pt]}
\renewcommand\cellgape{\Gape[4pt]}
\newcommand*\tick{\item[\Checkmark]}
\newcommand*\arrow{\item[$\Rightarrow$]}
\newcommand*\fail{\item[\XSolidBrush]}
\usepackage{minted} % for highlighting code sytax
\definecolor{LightGray}{gray}{0.9}
%\renewcommand*{\arraystretch}{2}
%\definecolor{LightGray}{gray}{0.9}

\setminted[text]{
	frame=lines, 
	breaklines,
	baselinestretch=1.2,
	bgcolor=LightGray,
%	fontsize=\small
}
\setminted[bash]{
%	frame=lines, 
	breaklines,
	baselinestretch=1.2,
	bgcolor=LightGray,
%	fontsize=\small
}
\setminted[python]{
	frame=lines, 
	breaklines, 
	linenos,
	baselinestretch=1.2,
%	bgcolor=LightGray,
%	fontsize=\small
}


\usepackage[square, numbers, comma, sort&compress]{natbib}  
\usepackage{verbatim}  % Needed for the "comment" environment to make LaTeX comments
% Allows "\bvec{}" and "\buvec{}" for "blackboard" style bold vectors in maths
\hypersetup{urlcolor=blue, colorlinks=true}  % Colours hyperlinks in blue, but this can be distracting if there are many links.

\begin{document}
	\maketitle
	
	
	\setstretch{1.0}
	\fancyhead{}
	\rhead{\thepage}
	\lhead{}
	
	\pagestyle{fancy}
	
	
\Declaration{

\addtocontents{toc}{\vspace{1em}}  

\vspace{1cm}

I hereby declare that the dissertation {\bf \textit{Real-time Automated, Scalable Data Integration Platform for Big Data Analytics}} submitted by me to the School of Computing Science and Engineering, VIT University Chennai, 600 127 in partial fulfillment of the requirements for the award of {\bf Master of Technology} in {\bf Computer Science \& Engineering with specialization in Cloud Computing} is a bona-fide record of the work carried out by me under the supervision of{\bf\textit{ Pradeep KV}}. \\
I further declare that the work reported in this dissertation, has not been submitted and will not be submitted, either in part or in full, for the award of any other degree or diploma of this institute or of any other institute or University.

\vspace{1cm}

Sign:\\
\rule[1em]{25em}{0.5pt}  % This prints a line for the signature

Name \& Reg. No.:\\
\rule[1em]{25em}{0.5pt}  % This prints a line for the signature
 
Date: \\
\rule[1em]{25em}{0.5pt}  % This prints a line to write the date
\thispagestyle{empty} 
}

\clearpage  % Abstract ended, start a new page

	
	
\Certificate{
\addtocontents{toc}{\vspace{0.5em}}  % Add a gap in the Contents, for aesthetics

This is to certify that the dissertation entitled {\bf \textit{\thesubject}} 
submitted by {\bf \textit{\theauthor}  
(Roll No. \therollno)} 
to Nirma University Ahmedabad, in partial fulfullment of the requirement for the award of
the degree of {\bf Master of Technology} in
{\bf Computer Science \& Engineering with specialization in Computer Science \& Engineering}  
is a bona-fide work carried out under my supervision.
The dissertation fulfills the requirements as per the regulations of this
University and in my opinion meets the necessary standards for submission.
The contents of this dissertation have not been submitted and will not be submitted
either in part or in full, for the award of any other degree or diploma
and the same is certified.

\vspace{1cm}
%
%   Spaces for signatures 
%
\begin{center}
\begin{tabular}{llllllll}
%
\multicolumn{2}{l}{\bf Supervisor}    
& & & & &\multicolumn{2}{l}{\bf Program Chair}   \\
%
     &  &  &  &  &     &      &    \\
%
Signature: &....................    & &  & &  & Signature:& ....................    \\
%
    &   & &  &  &  &    &    \\
%
Name:& .................... 
\qquad\quad   &  & &  &  &
 Name:& .................... \\
%
    &    &  & & &  &   &    \\
%
Date: &   & & &  &  & Date: &  \\
%
\end{tabular}
\\[0.75cm]

\end{center} 


\begin{tabular}{ll}

\bf{Examiner} & 
\\ & 
\\ Signature: & .................... 
\\  & 
\\ Name: & ....................      
\\ & 
\\ Date: & 
\\

\end{tabular}
 


\begin{flushright}{(Seal of the School)}\end{flushright}
\thispagestyle{empty}
}
\thispagestyle{empty}
\clearpage  

	
	
\addtotoc{Abstract} 
\abstract{
\addtocontents{toc}{\vspace{1em}}  

Intel System on a Chip (\gls{soc}) features a new set of Intel Intellectual Property (\gls{ip}) for every generation. \gls{bios} involves development of major individual components such as Processor, Graphics/Memory Controller, Input/Output Controller hub, System Monitor/Management Bus, Direct Media Interface, SATA/IDE/USB, Peripheral Component Interconnect (\gls{pci}), Voltage Regulator and Advanced Configuration and Power Interface (\gls{acpi}) for every Intel System on a Chip (\gls{soc}). Duration of every iteration of the development of any such components takes a much longer duration to build the BIOS binary and executing on hardware/\gls{soc} and verifying the functional work flow from logs. Even for some minor changes such as changing setup option value takes a high amount of time for iteration which is indeed slowing down the process of release and development of the Intel Intellectual Property (\gls{ip}). Major aim of this project will be to introduce a framework which can be used to reduce the cost of each build and test iteration for product development.

}

\clearpage  
	
	
\acknowledgements{
\addtocontents{toc}{\vspace{1em}} 

is this page needed ???????
}

\clearpage  % End of the Acknowledgements
%% ----------------------------------------------------------------

	\pagestyle{fancy}  
	
	\lhead{\emph{Contents}}  
	\tableofcontents  % Write out the Table of Contents

	\section{Introduction}

\subsection{Legacy \gls{bios} and \gls{uefi}}

\paragraph{\gls{bios}} is the dominant standard which defines a firmware interface.

"Legacy" (as in Legacy \gls{bios}), in the context of firmware specifications, refer to an older, widely used specification. Major responsibility of \gls{bios} is to set up the hardware, load and start an \gls{os}. When the system boots, the BIOS initializes and identifies system devices including video display card, mouse, hard disk drive, keyboard, solid state drive and other hardware followed by locating software held on a boot device i.e. a hard disk or removable storage such as CD/DVD or USB and loads and executes that software, giving it control of the computer. This process is also referred to as "booting" or "boot strapping".

\subsubsection{Background of Legacy \gls{bios}}
In 1980s, IBM developed the personal computer with a 16-bit BIOS with the aim of ending the BIOS after the first 250,000 products. Legacy BIOS is based upon Intel's original 16-bit architecture, ordinarily referred to as  "8086" architecture. And as technology advanced, Intel extended that 8086 architecture from 16 to 32-bit.
Legacy BIOS is able to run different \gls{os}, such as MS-DOS, equally well on systems other than IBM. Additionally, Legacy BIOS has a defined OS-independent interface for hardware that enables interrupts to communicate with video, disk and keyboard services along with the BIOS ROM loader and bootstrap loader, to name a few.

Use of legacy BIOS is diminishing and is expected to be phased out in new systems by the year 2020.

\subsection{Unified Extensible Firmware Interface (\gls{uefi})}
\gls{uefi} was developed as a replacement for legacy BIOS to streamline the booting process, and act as the interface between a operating system and its platform firmware. It not only replaces most BIOS functions, but also offers a rich extensible pre-OS environment with advanced boot and runtime services.
Unified Extensible Firmware Interface (\gls{uefi}) is grounded in Intel's initial Extensible Firmware Interface (EFI) specification 1.10, which defines a software interface between an operating system and platform firmware. The UEFI architecture allows users to execute applications on a command line interface. It has intrinsic networking capabilities and is designed to work with multi-processors (MP) systems.

\begin{figure}[h]
	\includegraphics[width=\linewidth]{uefi_board_of_directors}
	\caption{Board of Directors of UEFI Forum}\label{fig:introduction-uefi-board-of-directors}
\end{figure}

The UEFI Forum board of directors consists of representatives from 11 industry leaders as described in Figure \ref{fig:introduction-uefi-board-of-directors}. These involved organizations work to ensure that the UEFI specifications meet industry needs.

UEFI uses a different interface for boot services and runtime services but UEFI does not specify how "Power On Self Test" (POST) and Setup are implemented - those are BIOS' primary functions.

\subsubsection{\gls{uefi}'s Role in boot process}

During the boot process, UEFI speaks to the operating system loader and acts as the interface between the operating system and the BIOS.


\subsection{Comparing of Legacy \gls{bios} and \gls{uefi}}



\subsection{Advanced Configuration and Power Interface (\gls{acpi})}
The \say{ACPI Component Architecture (\gls{acpica})} is an implementation of a group of software components according to the ACPI specification. It is created with the goal of isolating operating system dependencies to a relatively small translation or conversion layer (the OS Services Layer). This makes the bulk of the ACPICA code independent of any individual operating system.so it can used for new operating systems with no source changes within the ACPICA code itself.

Tthe architecture include below component:
\begin{itemize}
  \item \gls{acpica} Subsystem - independent of OS and kernel which serves the primal ACPI services like the AML interpreter and management of namespace.
  \item \gls{acpica} Subsystem - independent of OS and OS Services Layer for every host OS to serve OS support.
  \item The ASL compiler/disassembler for translating the source code from ASL to AML and also disassembling the ASL source code from the binary ACPI tables if exists.
  \item Many ACPI utilities for running the interpreter in level 3 user space taking out the binary ACPI tables residing in the output result of ACPI Dump utility along with translating ACPICA source code to output format of Linux/Unix.
\end{itemize}

Figure \ref{fig:introduction-acpi-component-architecture} portrays the ACPICA subsystem in relation with the device driver(s), host OS, and the ACPI hardware.

\begin{figure}[!htbp]
	\centering
	\includegraphics[width=0.8\linewidth]{introduction/acpi-component-architecture}
	\caption{The \gls{acpi} Component Architecture}\label{fig:introduction-acpi-component-architecture}
\end{figure}

\subsubsection{Overview of \gls{acpica} Subsystem}
The \say{\gls{acpica} Subsystem} develops the basic primal aspects of the ACPI specification. Includes an AML parser/interpreter, ACPI table and device support, ACPI namespace management, and event handling. As the ACPICA subsystem serves the lower level services for system, it also involves low-level services of OS like memory management, scheduling, synchronization and I/O.

To allow the ACPICA Subsystem to easily link between any operating system that engage such services, an Operating System Services Layer transforms ACPICA-to-OS requests inside the system calls publicized by the host OS. This OS Services Layer is the one and only element of the ACPICA which pertains source code which is limited to a particular host OS.

%\subsubsection{OS-independent ACPICA Subsystem}
%The OS-independent ACPICA Subsystem supplies the major building blocks or subcomponents that are required for all ACPI implementations — including an AML interpreter, a namespace manager, ACPI event and resource management, and ACPI hardware support.
%
%One of the goals of the ACPICA Subsystem is to provide an abstraction level high enough such
%that the host operating system does not need to understand or know about the very low-level ACPI
%details. For example, all AML code is hidden from the host. Also, the details of the ACPI hardware
%are abstracted to higher-level software interfaces.
%
%The ACPICA Subsystem implementation makes no assumptions about the host operating system or environment. The only way it can request operating system services is via interfaces provided by the OS Services Layer.
%
%The primary user of the services provided by the ACPICA Subsystem are the host OS device drivers and power/thermal management software.
%
\subsubsection{Operating System Services Layer \gls{osl}}
\say{OS Services Layer (OSL)} act as a request translation service for host os from OS-independent ACPICA subsystem. The OSL develops a common subset for interfaces of OS service by utilizing the primitives usable from host OS.

The OSL has to be developed afresh for each and every supported host OS. There exists only one ACPICA Subsystem which OS-independent but there has to be a different OSL for each OS backed by the ACPICA.

The whole ACPICA in relation to the host OS is portrayed in Figure \ref{fig:introduction-acpica-subsystem-architecture}

\begin{figure}[!htbp]
	\centering
	\includegraphics[width=0.7\linewidth]{introduction/acpica-subsystem-architecture}
	\caption{ACPICA Subsystem Architecture}\label{fig:introduction-acpica-subsystem-architecture}
\end{figure}

\subsubsection{\gls{acpica} Subsystem Interaction}
ACPICA Subsystem develops a subset of external interface links that could directly summoned via host OS. These Acpi interfaces serve the literal ACPI services for host. When OS services are needed while servicing of request of an ACPI the Subsystem makes oblique request to host OS through the fixed AcpiOs interfaces. 

\begin{figure}[!htbp]
  \centering
  \includegraphics[width=0.7\linewidth]{introduction/acpi-interaction-between-the-architectural-components}
  \caption{Interaction between the Architectural Components}\label{fig:-introduction-acpi-interaction-between-the-architectural-components}
\end{figure}

Figure \ref{fig:-introduction-acpi-interaction-between-the-architectural-components} portrays the kinship and fundamental interaction linking the diverse architectural modules by screening the control flow among them. Note that OS independent ACPICA Subsystem could never call the host OS directly and instead it has to make call(s) to the AcpiOs interfaces inside the OSL. This serves the ACPICA code as OS-independence.


	\chapter{Design and Architecture}\label{chapter-design-and-architecture}
\lhead{Chapter 2. \emph{Design and Architecture}}
\section{Design}\label{section-design}

\subsection{Design Overview of \gls{uefi}}
The design of UEFI is based on the following fundamental elements:

\begin{itemize}
	\item \textbf{Reuse of existing table-based interfaces} - In order to preserve investment in existing infrastructure support code, both within the OS and firmware, variety of existing specifications that are commonly implemented on platforms compatible with supported processor specifications must be implemented on platforms wishing to comply with the UEFI specification.
	\item \textbf{System partition} defines a partition and file system that are designed to allow safe sharing between multiple vendors, and for different purposes. The ability to include a separate, shareable system partition presents an opportunity to increase platform value-add without significantly growing the need for nonvolatile platform memory
	\item \textbf{Boot services} provide interfaces for devices and system functionality that
	can be used during boot time. Device access is abstracted through "handles" and
	"protocols". This facilitates reuse of investment in existing BIOS code by keeping
	underlying implementation requirements out of the specification without burdening the
	consumer accessing the device.
	\item \textbf{Runtime services} - A minimal set of runtime services is presented to ensure appropriate	abstraction of base platform hardware resources that may be needed by the OS during its	normal operations.
\end{itemize}

\begin{figure}[h]
	\centering
	\includegraphics[width=0.8\linewidth]{design/uefi-conceptual-overview}
	\caption{UEFI Conceptual Overview}\label{fig:design-uefi-conceptual-overview}
\end{figure}

\textbf{Error! Reference source not found} illustrates the interactions of the various components of an UEFI specification-compliant system that are used to accomplish platform and OS boot.

The platform firmware can retrieve the OS loader image from the System Partition. The
specification provides for a variety of mass storage device types including disk, CD-ROM, and
DVD as well as remote boot via a network. Through the extensible protocol interfaces, it is possible
to add other boot media types, although these may require OS loader modifications if they require
use of protocols other than those defined in this document.

Once started, the OS loader continues to boot the complete operating system. To do so, it
may use the EFI boot services and interfaces defined by this or other required specifications to
survey, comprehend, and initialize the various platform components and the OS software that
manages them. EFI runtime services are also available to the OS loader during the boot phase.

\subsubsection{UEFI Driver Goals}
The UEFI Driver Model has the following goals:
\begin{itemize}
	\item \textbf{Compatible} - Drivers conforming to this specification must maintain compatibility with the EFI and the UEFI Specification. This means that the UEFI Driver Model takes advantage of the extensibility mechanisms in the UEFI Specification to add the required
	functionality.
	\item \textbf{Simple} - Drivers that conform to this specification must be simple to implement and	simple to maintain. The UEFI Driver Model must allow a driver writer to concentrate on
	the specific device for which the driver is being developed. A driver should not be
	concerned with platform policy or platform management issues. These considerations
	should be left to the system firmware.
	\item \textbf{Scalable} - The UEFI Driver Model must be able to adapt to all types of platforms. These platforms include embedded systems, mobile, and desktop systems, as well as workstations and servers.
	\item \textbf{Flexible} - The UEFI Driver Model must support the ability to enumerate all the devices, or to enumerate only those devices required to boot the required OS. The minimum device
	enumeration provides support for more rapid boot capability, and the full device enumeration provides the ability to perform OS installations, system maintenance, or system diagnostics on any boot device present in the system.
	\item \textbf{Extensible} - The UEFI Driver Model must be able to extend to future bus types as they
	are defined.

	\item \textbf{Portable} - Drivers written to the UEFI Driver Model must be portable between platforms and between supported processor architectures.
	\item \textbf{Interoperable} - Drivers must coexist with other drivers and system firmware and must do so without generating resource conflicts.
	\item \textbf{Describe complex bus hierarchies} - The UEFI Driver Model must be able to describe a variety of bus topologies from very simple single bus platforms to very complex platforms
	containing many buses of various types.

	\item \textbf{Small driver footprint} - The size of executables produced by the UEFI Driver Model must be minimized to reduce the overall platform cost. While flexibility and extensibility
	are goals, the additional overhead required to support these must be kept to a minimum to
	prevent the size of firmware components from becoming unmanageable.
	\item \textbf{Address legacy option rom issues} - The UEFI Driver Model must directly address and	solve the constraints and limitations of legacy option ROMs. Specifically, it must be
	possible to build add-in cards that support both UEFI drivers and legacy option ROMs,
	where such cards can execute in both legacy BIOS systems and UEFI-conforming latforms, without modifications to the code carried on the card. The solution must provide an evolutionary path to migrate from legacy option ROMs driver to UEFI drivers.
\end{itemize}

\subsection{\gls{uefi}/\gls{pi} Firmware Images}
\gls{uefi} and \gls{pi} specifications define the standardized format for EFI firmware storage devices (FLASH or other non-volatile storage) which are abstracted into "Firmware Volumes". Build systems must be capable of processing files to create the file formats described by the \gls{uefi} and PI specifications. The tools provided as part of the \gls{edk2} BaseTools package process files compiled by third party tools, as well as text and Unicode files in order to create UEFI or PI compliant binary image files. In some instances, where UEFI or PI specifications do not have an applicable input file format, such as the Visual Forms Representation (VFR) files used to create PI compliant IFR content, tools and documentation have been provided that allows the user to write text files that are processed into formats specified by UEFI or PI specifications.

\begin{figure}[h]
	\centering
	\includegraphics[width=0.8\linewidth]{design/uefi-pi-firmware-image-creation}
	\caption{UEFI/PI Firmware Image Creation}\label{fig:design-uefi-pi-firmware-image-creation}
\end{figure}

A Firmware Volume (FV) is a file level interface to firmware storage. Multiple FVs may be present in a single FLASH device, or a single FV may span multiple FLASH devices. An FV may be produced to support some other type of storage entirely, such as a disk partition or network device. For more information consult the Platform Initialization Specification, Volume 3.
In all cases, an FV is formatted with a binary file system. The file system used is typically the Firmware File System (FFS), but other file systems may be possible in some cases. Hence, all modules are stored as "files" in the FV. Some modules may be "execute in place" (linked at a fixed address and executed from the ROM), while others are relocated when they are loaded into memory and some modules may be able to run from ROM if memory is not present (at the time of the module load) or run from memory if it is available.
Files themselves have an internally defined binary format. This format allows for implementation of security, compression, signing, etc. Within this format, there are one or more "leaf" images. A leaf image could be, for example, a PE32 image for a DXE driver.

Therefore, there are several layers of organization to a full UEFI/PI firmware image. These layers are illustrated below in Figure \ref{fig:design-uefi-pi-firmware-image-creation}. Each transition between layers implies a processing step that transforms or combines previously processed files into the next higher level. Also shown in Figure \ref{fig:design-uefi-pi-firmware-image-creation} are the reference implementation tools that process the files to move them between the different layers.

\begin{figure}[h]
	\centering
	\includegraphics[width=0.9\linewidth]{design/efi-application-creation}
	\caption{UEFI/PI Firmware Image Creation}\label{fig:design-efi-application-creation}
\end{figure}


In addition to creating images that initialize a complete platform, the build process also supports creation of stand-alone UEFI applications (including OS Loaders) and Option ROM images containing driver code. Figure \ref{fig:design-efi-application-creation}, below, shows the reference implementation tools and creation processes for both of these image types

The final feature that is supported by the EDK II build process is the creation of Binary Modules that can be packaged and distributed for use by other organizations. Binary modules do not require distribution of the source code. This will permit vendors to distribute UEFI images without having to release proprietary source code.

This packaging process permits creation of an archive file containing one or more binary files that are either Firmware Image files or higher (EFI Section files, Firmware File system files, etc.). The build process will permit inserting these binary files into the appropriate level in the build stages.

\subsection{Platform Initialization \gls{pi} Boot Sequence}
Platform Initialization \gls{pi} compliant system firmware has to support the six phases: 
\begin{enumerate}
	\item Security (\gls{sec}) Phase
	\item Pre-efi Initialization (\gls{pei}) Phase
	\item Driver Execution Environment (\gls{dxe}) Phase
	\item Boot device selection (\gls{bds}) Phase
	\item Run time (RT) services and After Life (AL) (transition from the OS back to the firmware) of system. 
\end{enumerate}
Figure \ref{fig:design-pi-boot-phases} describes the phases and transition in detail.

\begin{figure}[h]
	\includegraphics[width=\linewidth]{PI_Boot_Phases}
	\caption{\gls{pi} Boot Phases}\label{fig:design-pi-boot-phases}
\end{figure}

\subsection{Security (\gls{sec})}
The Security (SEC) phase is the initial phase in the PI Architecture and is liable for the following:
\begin{itemize}
	\item Handling restart events of all platform
	\item Creation of a temporary memory store
	\item Bringing the root of trust in the system
	\item Transit handoff information to next phase - the PEI Foundation
\end{itemize}
The security section may have the modules with source code scripted in assembly language. Hence, some \gls{edk2} module development environment (MDE) modules can consist of assembly code. During Occurrence of this, both Windows and GCC versions of assembly language code are served in different files.

\subsection{Pre-EFI Initialization (\gls{pei})}
The Pre-EFI Initialization (PEI) phase described in the PI Architecture specifications is invoked quite betimes in the boot period. Specifically, after about preliminary processing in the Security (SEC) phase, any machine restart event will invoke the PEI phase.
The PEI phase is designed to be developed in many parts and consists of:
\begin{itemize}
	\item PEI Foundation (core code)
	\item Pre-EFI Initialization Modules (specialized plug-ins)
\end{itemize}
The PEI phase at first operates with the platform in a developing state, holding only on-processor resources, such as the cache of processor for call stack, to dispatch the Pre-EFI Initialization Modules (PEIMs).

The PEI phase cannot assume the availability of amounts of memory (RAM) as DXE and hence PEI phase limits its support to the following:
\begin{itemize}
	\item Locating and validating PEIMs
	\item Dispatching PEIMs
	\item Facilitating communication between PEIMs
	\item Providing handoff data to later phases
\end{itemize}

These PEIMs are responsible for the following:
\begin{itemize}
	\item Initializing some permanent memory complement
	\item Characterizing the memory in Hand-Off Blocks (HOBs)
	\item Characterizing the firmware volume locations in HOBs
	\item Transit the control into next phase - the Driver Execution Environment (DXE) phase
\end{itemize}

Figure \ref{fig:design-pei-operation-diagram} shows a diagram describes the action carried out during the PEI phase

\begin{figure}[h]
	\centering
	\includegraphics[width=0.8\linewidth]{design/pei-operation-diagram}
	\caption{Diagram of PI Operations}\label{fig:design-pei-operation-diagram}
\end{figure}

\subsubsection{PEI Services}
The PEI Foundation institutes a system table for the PEI Services named as PEI Services Table that is viewable to all \gls{pei} Modules (PEIMs) in the system. A PEI Service is defined as a method, command or other potentiality manifested by the PEI Foundation when that service's initialization needs are met. As the PEI phase having no permanent memory available until almost the end of the phase, all the various types of services created during this phase (PEI phase) cannot be as enrich as those created during later phases. A pointer to PEI Services Table is sent into entry point of each PEIM's and also to part of each PEIM-to-PEIM Interface (PPI) because the location of PEI Foundation and its temporary memory is unknown at build time. 

The PEI Foundation provides the classes of services listed in Table \ref{table:design-pei-foundation-class-service}

\begin{table}[h]
	\centering
	\renewcommand*{\arraystretch}{2}
	\caption{Services provided by PEI Foundation Classes}\label{table:design-pei-foundation-class-service}
	\begin{tabular}{ l | p{9cm} }
		Service & Details
		\\ \hline \hline
		PPI Services & Manages PPIs to ease inter-module method calls between PEIMs. A database maintained in temporary RAM to track installed interfaces.
		\\ \hline
		Boot Mode Services & Manages the boot mode (S3, S5, diagnostics, normal boot, etc.)
		\\ \hline
		HOB Services & Creates data structures (Hand-off-blocks) that are used to convey information to the next phase 
		\\ \hline
		Firmware Volume Services & Finds PEIMs and along with that other firmware files in the firmware volumes
		\\ \hline
		PEI Memory Services & provides a collection of memory management services (to be used before and after permanent memory to discovered)
		\\ \hline
		Status Code Services & Provides general progress and error code reporting services (i.e. port 080h or a serial port for text output for debug)
		\\ \hline
		Reset Services & Provides a common means to aid initializing warm or cold restart of the system
		\\ \hline
	\end{tabular}

\end{table}


\subsubsection{PEI Foundation}
The PEI Foundation is the entity that carried outs following activity:
\begin{itemize}
	\item Dispatching of Pre-EFI initialization modules (PEIMs)
	\item Maintaining the boot mode
	\item Initialization of permanent memory
	\item Invoking the DXE loader 
\end{itemize}
The PEI Foundation written to be portable across all the various platforms architecture of a given instruction-set. i.e. A binary for IA-32 (32-bit Intel architecture) works across all Pentium processors and similarly Itanium processor family work across all Itanium processors.

Irrespective of the processor micro architecture, the set of services uncovered by the PEI Foundation should be the same. This consistent surface area around the PEI Foundation allows PEIMs to be written in the \verb|C programming language| and compiled across any micro architecture.

\subsection{PEI Dispatcher}
The PEI Dispatcher is basically a state machine which is implemented in the PEI Foundation. The PEI Dispatcher evaluates the dependency expressions in Pre-EFI initialization modules (PEIMs) that are lying in the \gls{fv}s being examined.

Dependency expressions are coherent combinations of PEIM-to-PEIM Interfaces (PPIs). These expressions distinguish the PPIs that must be available for use before a given PEIM can be invoked. The PEI Dispatcher references the PPI database in the PEI Foundation to conclude which PPIs have to be installed and evaluate the dependency expression for the PEIM. If PPI has already been installed then dependency expression will evaluate to \verb|TRUE|, which notifies  PEI Dispatcher it can run PEIM. At this stage, the PEI Foundation handovers control to the PEIM with \verb|TRUE| dependency expression. 

The PEI Dispatcher will exit Once the PEI Dispatcher has examined and evaluated all of the PEIMs in all of the uncovered firmware volumes and no more PEIMs can be dispatched (i.e. the dependency expressions (\gls{depex}) do not evaluate from \verb|FALSE| to \verb|TRUE|). At this stage, the PEI Dispatcher cannot invoke any additional PEIMs. The PEI Foundation then takes back control from the PEI Dispatcher and calls the \verb|DXE IPL PPI| to navigate control to the DXE phase of execution.

\subsection{Drive Execution Environment (\gls{dxe})}
Before the DXE phase, the Pre-EFI Initialization (PEI) phase is held responsible for initializing permanent memory in the platform. Hence, DXE phase can be loaded and executed. At the very end of the PEI phase, state of the system is handed over to the DXE phase through Hand-Off Blocks (list of position independent data structures). 

There are three components in the DXE phase:
\begin{enumerate}
	\item DXE Foundation
	\item DXE Dispatcher
	\item A set of DXE Drivers
\end{enumerate}

\subsection{Boot Device Selection (\gls{bds})}
The BDS Architectural Protocol has part of implementation of the Boot Device Selection (BDS) phase. After evaluation of all of the DXE drivers dependencies, DXE drivers with satisfied dependencies are loaded and executed by the DXE Dispatcher, the DXE Foundation will pass the control to the BDS Architectural Protocol. The BDS phase held responsible for the following:

\begin{itemize}
	\item Initializing console devices
	\item Loading device drivers
	\item Attempt of loading and executing boot selections
\end{itemize}

\subsection{Transient System Load (TSL) and Runtime (RT)}
Primarily the OS vendor provides boot loader known as The Transient System Load (TSL). TSL and Runtime Services (RT) phases may allow access to persistent content, via UEFI drivers and applications. Drivers in this category include PCI Option ROMs.

\subsection{After Life (AL)}
The After Life (AL) phase contains the persistent UEFI drivers used to store the state of the system during the OS systematically shutdown, sleep, hibernate or restart processes.



\subsection{Generic Build Process}
All code starts out as either C sources and header files, assembly sources and header files, UCS-2 HII strings in Unicode files, Virtual Forms Representation files or binary data (native instructions, such as microcode) files. Per the UEFI and PI specifications, the C and Assembly files must be compiled and linked into PE32/PE32+ images.
While some code is designed to execute only from ROM, most UEFI/PI modules are written to be relocate-able. These are written and built different. For example, Execute In Place (XIP) module code is written and compiled to run from ROM, while the majority of the code is written and compiled to execute from memory, which requires that the code be relocate able.
Some modules may also permit dual mode, where it will execute from memory only if memory is available, otherwise it will execute from ROM. Additionally, modules may permit dual access, such as a driver that contains both PEI and DXE implementation code. Code is assembled or compiled, then linked into PE32/PE32+ images, the relocation section may or may not be stripped and an appropriate header will replace the PE32/PE32+ header. Additional processing may remove more non-essential information, generating a Terse (TE) image.
The binary executables are converted into EFI firmware file sections. Each module is converted into an EFI Section consisting of an Section header followed by the section data (driver binary).

\subsubsection{EFI Section Files}
he general section format for sections less than 16MB in size is shown in Figure \ref{fig:design-general-efi-section-format}. Figure \ref{fig:design-general-efi-section-format-large} shows the section format for sections 16MB or larger in size using the extended length field.

\begin{figure}[h]
	\centering
	\includegraphics[width=0.7\linewidth]{design/general-efi-section-format-large}
	\caption{General EFI Section Format for large size Sections(greater then 16 MB)}\label{fig:design-general-efi-section-format-large}
\end{figure}


\begin{figure}[h]
	\centering
	\includegraphics[width=0.7\linewidth]{design/general-efi-section-format}
	\caption{General EFI Section Format (less then 16 MB)}\label{fig:design-general-efi-section-format}
\end{figure}



\subsection{Cross Compatibility of CPUs}
Whenever customer try to change the default Intel motherboard CPU with different Intel silicon chip which won’t works. The specific CPU Chip initialization varies for each generation. So, the board designs should be designed is such a way that specific generation CPU should support., if we change the CPU with a different Intel Board it will not even boot, because the BIOS doesn't support for other Silicon Initialization for other CPUs.

So, we are integrating the runtime detection of the silicon during the Pre-Extensible Firmware Initialization (\gls{pei}) phase. So, within single Integrated Firmware Image (\gls{ifwi}) should support the Multi Generation CPUs which is never tried before.

Each silicon has a fixed register from which the CPU generation can be identified., so the BIOS should read that register and program in such a way the is CPU1 is in Platform it should support the CPU1 Features like PCIe, Graphics \& DMI., if the CPU1 is replaced with CPU2 then it should support the CPU2 speed. That should be taken care by the BIOS.

\begin{figure}[h]
	\centering
	\includegraphics[width=0.7\linewidth]{design/cross-compatibility-design}
	\caption{Cross Compatibility Design}\label{fig:design-cross-compatibility-design}
\end{figure}

Figure \ref{fig:design-cross-compatibility-design} shows the general view of the Cross Compatibility of CPUs.

BIOS is the part of Integrated Firmware Image which resides at the End of the Binary table. The CPU swap can only occur in Specially designed Intel Designed Board only. Mainly because for each and every feature it required some hardware(H/W) requirements. If that H/W requirement not present. Then It will boot but doesn't support the Maximum speed.


\begin{figure}[h]
	\centering
	\includegraphics[width=0.7\linewidth]{design/bios-support-for-cross-compatibility}
	\caption{BIOS Support for Cross Compatibility}\label{fig:design-bios-support-for-cross-compatibility}
\end{figure}

Figure \ref{fig:design-bios-support-for-cross-compatibility} shows the BIOS role for identifying the CPUs during PEI phase.

As the number of Feature increases in the Silicon BIOS size also increases, usually the BIOS size varies from Platform to Platform and CPU to CPU., as we are integrating the Compatibility the BIOS size obviously increases. 

The structure of \gls{ifwi} is Shown in Figure \ref{fig:design-integrated-firmware-image}

\begin{figure}[h]
	\centering
	\includegraphics[width=0.7\linewidth]{design/integrated-firmware-image}
	\caption{Integrated Firmware Image}\label{fig:design-integrated-firmware-image}
\end{figure}




	\printglossary[type=main]
\end{document}