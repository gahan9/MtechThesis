\begin{frame}[allowframebreaks]{Major Components}
    \begin{redblock}{TianoCore}
        The community supporting an open source implementation of the Unified Extensible Firmware Interface (UEFI).
    \end{redblock}
    
    \begin{greenblock}{UEFI \footnote{Unified Extensible Firmware Interface}}
        Specifies the layer between an operating system and the platform firmware.
    \end{greenblock}
    
    \begin{blueblock}{EDK II}
    A modern, feature-rich, cross-platform firmware development environment for the UEFI and UEFI Platform Initialization (PI) specifications.
    \end{blueblock}

    \begin{redblock}{ACPI \footnote{Advanced Configuration and Power Interface} Component Architecture}
        A major goal of the architecture is to isolate all operating system dependencies to a relatively small translation or conversion layer (the OS Services Layer) so that the bulk of the ACPICA code is independent of any individual operating system.
    \end{redblock}


    \begin{greenblock}{PCIe \footnote{Peripheral Component Interconnect Express}}
        A major goal of the architecture is to isolate all operating system dependencies to a relatively small translation or conversion layer (the OS Services Layer) so that the bulk of the ACPICA code is independent of any individual operating system.
    \end{greenblock}

\end{frame}

\begin{frame}{PC Architecture of Intel}
    \begin{itemize}
        \item Every Intel architecture hardware platform includes 2 major components:
            \begin{itemize}
                \item Microprocessor Chip
                \item Companion Chip (PCH) 
            \end{itemize}
        \item Previously, Intel Processors were paired with 2 companion chips: North Bridge \& South Bridge
        \item Now, the functions of the north bridge are usually included in the processor itself.
        \item South Bridge is replaced by the PCH (Platform Control Hub)
    \end{itemize}
\end{frame}
