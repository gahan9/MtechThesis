%%%%%%%%%%%%%%%%%%%%%%%%%%%%%%%%%%%%%%%%%
% Journal Article
% 
% Gahan M. Saraiya
% 18MCEC10
%
% References
% ==========
% 
%%%%%%%%%%%%%%%%%%%%%%%%%%%%%%%%%%%%%%%%%
\documentclass[a4paper,12pt,oneside]{Thesis}
%\usepackage[a4paper]{geometry}
\usepackage[utf8]{inputenc}
\usepackage[english]{babel}

\usepackage{graphicx}
\graphicspath{ {./assets/} {./pics/} {./eps/} {./figures/}}


%%%%%%%%%%%%%%%%%%%%%%%%%%%%%%%%%%%%%%%%%%%%%%%%%%
%% COLOR DEFINITIONS
%%%%%%%%%%%%%%%%%%%%%%%%%%%%%%%%%%%%%%%%%%%%%%%%%%
\usepackage[svgnames]{xcolor} % Enabling mixing colors and color's call by 'svgnames'
%%%%%%%%%%%%%%%%%%%%%%%%%%%%%%%%%%%%%%%%%%%%%%%%%%
\definecolor{MyColor1}{rgb}{0.2,0.4,0.6} %mix personal color
\newcommand{\textb}{\color{Black} \usefont{OT1}{lmss}{m}{n}}
\newcommand{\blue}{\color{MyColor1} \usefont{OT1}{lmss}{m}{n}}
\newcommand{\blueb}{\color{MyColor1} \usefont{OT1}{lmss}{b}{n}}
\newcommand{\red}{\color{LightCoral} \usefont{OT1}{lmss}{m}{n}}
\newcommand{\green}{\color{Turquoise} \usefont{OT1}{lmss}{m}{n}}
%%%%%%%%%%%%%%%%%%%%%%%%%%%%%%%%%%%%%%%%%%%%%%%%%%




%%%%%%%%%%%%%%%%%%%%%%%%%%%%%%%%%%%%%%%%%%%%%%%%%%
%% FONTS AND COLORS
%%%%%%%%%%%%%%%%%%%%%%%%%%%%%%%%%%%%%%%%%%%%%%%%%%
%    SECTIONS
%%%%%%%%%%%%%%%%%%%%%%%%%%%%%%%%%%%%%%%%%%%%%%%%%%
\usepackage{titlesec}
\usepackage{sectsty}
%%%%%%%%%%%%%%%%%%%%%%%%
%%set section/subsections HEADINGS font and color
%\sectionfont{\color{MyColor1}}  % sets colour of sections
%\subsectionfont{\color{MyColor1}}  % sets colour of sections
%
%%set section enumerator to arabic number (see footnotes markings alternatives)
\renewcommand\thesection{\arabic{section}.} %define sections numbering
\renewcommand\thesubsection{\thesection\arabic{subsection}} %subsec.num.
%
%%define new section style
%\newcommand{\mysection}{
%    \titleformat{\section} [runin] {\usefont{OT1}{lmss}{b}{n}\color{MyColor1}} 
%    {\thesection} {3pt} {} } 

% Glossaries build
\usepackage[acronym]{glossaries}
\makeglossaries
\loadglsentries{sections/glossaries}
%\makenoidxglossaries


\usepackage{longtable}

%----------------------------------------------------------------------------------------
%       DATE FORMAT
%----------------------------------------------------------------------------------------
\usepackage{datetime}
\newdateformat{monthyeardate}{\monthname[\THEMONTH], \THEYEAR}
%----------------------------------------------------------------------------------------

%\titleformat{\section}[block]{\large\scshape\centering}{\thesection.}{1em}{} % Change the look of the section titles
%\titleformat{\subsection}[block]{\large}{\thesubsection.}{1em}{} % Change the look of the section titles
%\newcommand{\horrule}[1]{\rule{\linewidth}{#1}} % Create horizontal rule command with 1 argument of height
%\usepackage{fancyhdr} % Headers and footers
%\pagestyle{fancy} % All pages have headers and footers
%\fancyhead{} % Blank out the default header
%\fancyfoot{} % Blank out the default footer



%%%%%%%%%%%%%%%%%%%%%%%%%%%%%%%%%%%%%%%%%%%%%%%%%%
%		CAPTIONS
%%%%%%%%%%%%%%%%%%%%%%%%%%%%%%%%%%%%%%%%%%%%%%%%%%
\usepackage{caption}
%\usepackage{subcaption}
%%%%%%%%%%%%%%%%%%%%%%%%%
%\captionsetup[figure]{labelfont={color=Turquoise}}

%%%%%%%%%%%%%%%%%%%%%%%%%%%%%%%%%%%%%%%%%%%%%%%%%%
%		!!!EQUATION (ARRAY) --> USING ALIGN INSTEAD
%%%%%%%%%%%%%%%%%%%%%%%%%%%%%%%%%%%%%%%%%%%%%%%%%%
%using amsmath package to redefine eq. numeration (1.1, 1.2, ...) 
%%%%%%%%%%%%%%%%%%%%%%%%

%\renewcommand{\theequation}{\thesection\arabic{equation}}
%
%%set box background to grey in align environment 
%\usepackage{etoolbox}% http://ctan.org/pkg/etoolbox
%\makeatletter
%\patchcmd{\@Aboxed}{\boxed{#1#2}}{\colorbox{black!15}{$#1#2$}}{}{}%
%\patchcmd{\@boxed}{\boxed{#1#2}}{\colorbox{black!15}{$#1#2$}}{}{}%
%\makeatother

%%%%%%%%%%%%%%%%%%%%%%%%%%%%%%%%%%%%%%%%%%%%%%%%%%

% -------------------------------------------------------------------------------
% *** FLOWCHART AND GRAPHS PACKAGES ***
% -------------------------------------------------------------------------------
\usepackage{tikz}
\usepackage{pgfplots}
\usepackage{neuralnetwork}
\pgfplotsset{compat=1.5.1}
\usetikzlibrary{snakes, arrows, shapes, shapes.geometric, calc, automata, positioning}
\tikzstyle{startstop} = [rectangle, rounded corners, minimum height=1cm, minimum width=2cm, 
text centered, trapezium stretches=true, draw=black, 
%fill=red!30
]

\tikzstyle{io} = [trapezium, trapezium left angle=70, trapezium right angle=110, minimum width=3cm, trapezium stretches=true, minimum height=1cm, text centered, draw=black, 
%fill=blue!30
]
\tikzstyle{process} = [rectangle, minimum width=3cm, minimum height=1cm, text centered, draw=black, trapezium stretches=true, 
%fill=orange!30
]
\tikzstyle{decision} = [diamond, minimum width=3cm, minimum height=1cm, text centered, draw=black, trapezium stretches=true, 
%fill=green!30
]
\tikzstyle{arrow} = [thick,->,>=stealth]
\pgfplotsset{every axis/.append style={tick label style={/pgf/number format/fixed},font=\scriptsize,ylabel near ticks,xlabel near ticks,grid=major}}
\tikzset{%
	every neuron/.style={
		circle,
		draw,
		minimum size=1cm
	},
	neuron missing/.style={
		draw=none, 
		scale=4,
		text height=0.333cm,
		execute at begin node=\color{black}$\vdots$
	},
}
% -------------------------------------------------------------------------------

% -------------------------------------------------------------------------------
% *** INDIAN RUPEE SYMBOL ***
% -------------------------------------------------------------------------------
\usepackage{tfrupee} 
% ------------ INDIAN RUPEE SYMBOL END ---------------------------


% -------------------------------------------------------------------------------
% SET UP MATH Commands and configs
% -------------------------------------------------------------------------------


\usepackage{amsmath, amssymb, amsfonts, amsthm, fouriernc, mathtools}
% mathtools for: Aboxed (put box on last equation in align envirenment)
%\usepackage{microtype} %improves the spacing between words and letters

\usepackage{theorem}
% Special Matrix
\newenvironment{spmatrix}[1]
{\def\mysubscript{#1}\mathop\bgroup\begin{bmatrix}}
	{\end{bmatrix}\egroup_{\textstyle\mathstrut\mysubscript}}
% Adding explaination below equation terms
\newcommand{\explain}[2]{\underbrace{#1}_{\parbox{\widthof{$#1$}}{\tiny#2}}}
%\newcommand{\explain}[2]{\underbrace{#1}_{\parbox{\widthof{#1}}{\footnotesize\raggedright #2}}}

%%%%%%%%%%%%%%%%%%%%%%%%%%%%%%%%%%%%%%%%%%%%%%%%%%
%% DESIGN CIRCUITS
%%%%%%%%%%%%%%%%%%%%%%%%%%%%%%%%%%%%%%%%%%%%%%%%%%
%\usepackage[siunitx, american, smartlabels, cute inductors, europeanvoltages]{circuitikz}
%%%%%%%%%%%%%%%%%%%%%%%%%%%%%%%%%%%%%%%%%%%%%%%%%%



%\makeatletter
%\let\reftagform@=\tagform@
%\def\tagform@#1{\maketag@@@{(\ignorespaces\textcolor{red}{#1}\unskip\@@italiccorr)}}
%\renewcommand{\eqref}[1]{\textup{\reftagform@{\ref{#1}}}}
%\makeatother
%\usepackage{hyperref}
%\hypersetup{colorlinks=true}

% to allow adding line break in table cell

\usepackage{makecell}
\hypersetup{%
  colorlinks=true,
  linkcolor=black,
  filecolor=black,
  linkbordercolor={0 0 1}
}
 
\renewcommand\lstlistingname{Algorithm}
\renewcommand\lstlistlistingname{Algorithms}
\def\lstlistingautorefname{Alg.}

\lstdefinestyle{Python}{
    language        = Python,
    frame           = lines, 
    basicstyle      = \footnotesize,
    keywordstyle    = \color{blue},
    stringstyle     = \color{green},
    commentstyle    = \color{red}\ttfamily
}

%\setlength{\parindent}{0.0in}
%\setlength{\parskip}{0.05in}

%\date{March, 2019}

%%%%%%%%%%%%%%%%%%%%%%%%%%%%%%%%%%%%%%%%%%%%%%%%%%
%----------------------------------------------------------------------------------------
%       SET HEADER AND FOOTER
%----------------------------------------------------------------------------------------
\newcommand\theauthor{Gahan Saraiya}
\newcommand\therollno{18MCEC10}
\newcommand\thesubject{Generic IP independent BIOS Signing and Parsing}
\newcommand\theinstitute{Institute of Technology}
\newcommand\theuniversity{Nirma University}
\newcommand\thedegree{Master of Technology}
\newcommand\thebranch{Computer Science \& Engineering}
%\newcommand{\HRule}{\rule{\linewidth}{0.8mm}}
\renewcommand{\footrulewidth}{0.4pt}% default is 0pt
%\fancyhead[C]{\theinstitute, \theuniversity $\bullet$ \monthyeardate\today} % Custom header text
%\fancyfoot[LE,LO]{\thesubject}
%\fancyfoot[RO,LE]{Page \thepage} % Custom footer text
%----------------------------------------------------------------------------------------

\renewcommand\theadalign{bc}
\renewcommand\theadfont{\bfseries}
\renewcommand\theadgape{\Gape[4pt]}
\renewcommand\cellgape{\Gape[4pt]}
\newcommand*\tick{\item[\Checkmark]}
\newcommand*\arrow{\item[$\Rightarrow$]}
\newcommand*\fail{\item[\XSolidBrush]}
\usepackage{minted} % for highlighting code sytax
\captionsetup[listing]{position=top}

\definecolor{LightGray}{gray}{0.9}
%\renewcommand*{\arraystretch}{2}
%\definecolor{LightGray}{gray}{0.9}

\setminted[text]{
	frame=lines, 
	breaklines,
	baselinestretch=1.2,
	bgcolor=LightGray,
%	fontsize=\small
}
\setminted[bash]{
%	frame=lines, 
	breaklines,
	baselinestretch=1.2,
	bgcolor=LightGray,
%	fontsize=\small
}
\setminted[python]{
	frame=lines, 
	breaklines, 
	linenos,
	baselinestretch=1.2,
%	bgcolor=LightGray,
%	fontsize=\small
}

\setminted[c]{
	frame=lines, 
	breaklines, 
	linenos,
	baselinestretch=1.2,
%	bgcolor=LightGray,
%	fontsize=\small
}


\usepackage[square, numbers, comma, sort&compress]{natbib}  
\usepackage{verbatim}  % Needed for the "comment" environment to make LaTeX comments
% Allows "\bvec{}" and "\buvec{}" for "blackboard" style bold vectors in maths
%\hypersetup{urlcolor=blue, colorlinks=true}  % Colours hyperlinks in blue, but this can be distracting if there are many links.



%%%%%%%%%%%%%%%%%%%%%%%%%%%%%%%%%%%%%%%%%%%%%%%%%%
%% PREPARE TITLE
%%%%%%%%%%%%%%%%%%%%%%%%%%%%%%%%%%%%%%%%%%%%%%%%%%
\title{
%		\blue Project Report \\
%	\blueb \thesubject}
	\thesubject}
\author{\theauthor (18MCEC10)}
\date{\monthyeardate\today}

% -------------------------------------------------------------------
% BEGIN THE DOCUMENT ------------------------------------------------
% -------------------------------------------------------------------
\usepackage{atbegshi}% http://ctan.org/pkg/atbegshi
\AtBeginDocument{\AtBeginShipoutNext{\AtBeginShipoutDiscard}}
\begin{document}
\pagestyle{empty}
\maketitle
\newpage


\setstretch{1.0}
\fancyhead{}
\rhead{\thepage}
\lhead{}

\pagestyle{fancy}
\renewcommand{\thepage}{\roman{page}}


\Declaration{

\addtocontents{toc}{\vspace{1em}}  

\vspace{1cm}

I hereby declare that the dissertation {\bf \textit{Real-time Automated, Scalable Data Integration Platform for Big Data Analytics}} submitted by me to the School of Computing Science and Engineering, VIT University Chennai, 600 127 in partial fulfillment of the requirements for the award of {\bf Master of Technology} in {\bf Computer Science \& Engineering with specialization in Cloud Computing} is a bona-fide record of the work carried out by me under the supervision of{\bf\textit{ Pradeep KV}}. \\
I further declare that the work reported in this dissertation, has not been submitted and will not be submitted, either in part or in full, for the award of any other degree or diploma of this institute or of any other institute or University.

\vspace{1cm}

Sign:\\
\rule[1em]{25em}{0.5pt}  % This prints a line for the signature

Name \& Reg. No.:\\
\rule[1em]{25em}{0.5pt}  % This prints a line for the signature
 
Date: \\
\rule[1em]{25em}{0.5pt}  % This prints a line to write the date
\thispagestyle{empty} 
}

\clearpage  % Abstract ended, start a new page


\Certificate{
\addtocontents{toc}{\vspace{0.5em}}  % Add a gap in the Contents, for aesthetics

This is to certify that the dissertation entitled {\bf \textit{\thesubject}} 
submitted by {\bf \textit{\theauthor}  
(Roll No. \therollno)} 
to Nirma University Ahmedabad, in partial fulfullment of the requirement for the award of
the degree of {\bf Master of Technology} in
{\bf Computer Science \& Engineering with specialization in Computer Science \& Engineering}  
is a bona-fide work carried out under my supervision.
The dissertation fulfills the requirements as per the regulations of this
University and in my opinion meets the necessary standards for submission.
The contents of this dissertation have not been submitted and will not be submitted
either in part or in full, for the award of any other degree or diploma
and the same is certified.

\vspace{1cm}
%
%   Spaces for signatures 
%
\begin{center}
\begin{tabular}{llllllll}
%
\multicolumn{2}{l}{\bf Supervisor}    
& & & & &\multicolumn{2}{l}{\bf Program Chair}   \\
%
     &  &  &  &  &     &      &    \\
%
Signature: &....................    & &  & &  & Signature:& ....................    \\
%
    &   & &  &  &  &    &    \\
%
Name:& .................... 
\qquad\quad   &  & &  &  &
 Name:& .................... \\
%
    &    &  & & &  &   &    \\
%
Date: &   & & &  &  & Date: &  \\
%
\end{tabular}
\\[0.75cm]

\end{center} 


\begin{tabular}{ll}

\bf{Examiner} & 
\\ & 
\\ Signature: & .................... 
\\  & 
\\ Name: & ....................      
\\ & 
\\ Date: & 
\\

\end{tabular}
 


\begin{flushright}{(Seal of the School)}\end{flushright}
\thispagestyle{empty}
}
\thispagestyle{empty}
\clearpage  


\addtotoc{Abstract} 
\abstract{
\addtocontents{toc}{\vspace{1em}}  

Intel System on a Chip (\gls{soc}) features a new set of Intel Intellectual Property (\gls{ip}) for every generation. \gls{bios} involves development of major individual components such as Processor, Graphics/Memory Controller, Input/Output Controller hub, System Monitor/Management Bus, Direct Media Interface, SATA/IDE/USB, Peripheral Component Interconnect (\gls{pci}), Voltage Regulator and Advanced Configuration and Power Interface (\gls{acpi}) for every Intel System on a Chip (\gls{soc}). Duration of every iteration of the development of any such components takes a much longer duration to build the BIOS binary and executing on hardware/\gls{soc} and verifying the functional work flow from logs. Even for some minor changes such as changing setup option value takes a high amount of time for iteration which is indeed slowing down the process of release and development of the Intel Intellectual Property (\gls{ip}). Major aim of this project will be to introduce a framework which can be used to reduce the cost of each build and test iteration for product development.

}

\clearpage  

\acknowledgements{
\addtocontents{toc}{\vspace{1em}} 

is this page needed ???????
}

\clearpage  % End of the Acknowledgements
%% ----------------------------------------------------------------

\pagestyle{fancy}  

%\lhead{\emph{Contents}}  
\tableofcontents  % Write out the Table of Contents

\newpage
\listoffigures % \addcontentsline{toc}{chapter}{List of Figures}

\pagenumbering{arabic}

\section{Introduction}

\subsection{Legacy \gls{bios} and \gls{uefi}}

\paragraph{\gls{bios}} is the dominant standard which defines a firmware interface.

"Legacy" (as in Legacy \gls{bios}), in the context of firmware specifications, refer to an older, widely used specification. Major responsibility of \gls{bios} is to set up the hardware, load and start an \gls{os}. When the system boots, the BIOS initializes and identifies system devices including video display card, mouse, hard disk drive, keyboard, solid state drive and other hardware followed by locating software held on a boot device i.e. a hard disk or removable storage such as CD/DVD or USB and loads and executes that software, giving it control of the computer. This process is also referred to as "booting" or "boot strapping".

\subsubsection{Background of Legacy \gls{bios}}
In 1980s, IBM developed the personal computer with a 16-bit BIOS with the aim of ending the BIOS after the first 250,000 products. Legacy BIOS is based upon Intel's original 16-bit architecture, ordinarily referred to as  "8086" architecture. And as technology advanced, Intel extended that 8086 architecture from 16 to 32-bit.
Legacy BIOS is able to run different \gls{os}, such as MS-DOS, equally well on systems other than IBM. Additionally, Legacy BIOS has a defined OS-independent interface for hardware that enables interrupts to communicate with video, disk and keyboard services along with the BIOS ROM loader and bootstrap loader, to name a few.

Use of legacy BIOS is diminishing and is expected to be phased out in new systems by the year 2020.

\subsection{Unified Extensible Firmware Interface (\gls{uefi})}
\gls{uefi} was developed as a replacement for legacy BIOS to streamline the booting process, and act as the interface between a operating system and its platform firmware. It not only replaces most BIOS functions, but also offers a rich extensible pre-OS environment with advanced boot and runtime services.
Unified Extensible Firmware Interface (\gls{uefi}) is grounded in Intel's initial Extensible Firmware Interface (EFI) specification 1.10, which defines a software interface between an operating system and platform firmware. The UEFI architecture allows users to execute applications on a command line interface. It has intrinsic networking capabilities and is designed to work with multi-processors (MP) systems.

\begin{figure}[h]
	\includegraphics[width=\linewidth]{uefi_board_of_directors}
	\caption{Board of Directors of UEFI Forum}\label{fig:introduction-uefi-board-of-directors}
\end{figure}

The UEFI Forum board of directors consists of representatives from 11 industry leaders as described in Figure \ref{fig:introduction-uefi-board-of-directors}. These involved organizations work to ensure that the UEFI specifications meet industry needs.

UEFI uses a different interface for boot services and runtime services but UEFI does not specify how "Power On Self Test" (POST) and Setup are implemented - those are BIOS' primary functions.

\subsubsection{\gls{uefi}'s Role in boot process}

During the boot process, UEFI speaks to the operating system loader and acts as the interface between the operating system and the BIOS.


\subsection{Comparing of Legacy \gls{bios} and \gls{uefi}}



\subsection{Advanced Configuration and Power Interface (\gls{acpi})}
The \say{ACPI Component Architecture (\gls{acpica})} is an implementation of a group of software components according to the ACPI specification. It is created with the goal of isolating operating system dependencies to a relatively small translation or conversion layer (the OS Services Layer). This makes the bulk of the ACPICA code independent of any individual operating system.so it can used for new operating systems with no source changes within the ACPICA code itself.

Tthe architecture include below component:
\begin{itemize}
  \item \gls{acpica} Subsystem - independent of OS and kernel which serves the primal ACPI services like the AML interpreter and management of namespace.
  \item \gls{acpica} Subsystem - independent of OS and OS Services Layer for every host OS to serve OS support.
  \item The ASL compiler/disassembler for translating the source code from ASL to AML and also disassembling the ASL source code from the binary ACPI tables if exists.
  \item Many ACPI utilities for running the interpreter in level 3 user space taking out the binary ACPI tables residing in the output result of ACPI Dump utility along with translating ACPICA source code to output format of Linux/Unix.
\end{itemize}

Figure \ref{fig:introduction-acpi-component-architecture} portrays the ACPICA subsystem in relation with the device driver(s), host OS, and the ACPI hardware.

\begin{figure}[!htbp]
	\centering
	\includegraphics[width=0.8\linewidth]{introduction/acpi-component-architecture}
	\caption{The \gls{acpi} Component Architecture}\label{fig:introduction-acpi-component-architecture}
\end{figure}

\subsubsection{Overview of \gls{acpica} Subsystem}
The \say{\gls{acpica} Subsystem} develops the basic primal aspects of the ACPI specification. Includes an AML parser/interpreter, ACPI table and device support, ACPI namespace management, and event handling. As the ACPICA subsystem serves the lower level services for system, it also involves low-level services of OS like memory management, scheduling, synchronization and I/O.

To allow the ACPICA Subsystem to easily link between any operating system that engage such services, an Operating System Services Layer transforms ACPICA-to-OS requests inside the system calls publicized by the host OS. This OS Services Layer is the one and only element of the ACPICA which pertains source code which is limited to a particular host OS.

%\subsubsection{OS-independent ACPICA Subsystem}
%The OS-independent ACPICA Subsystem supplies the major building blocks or subcomponents that are required for all ACPI implementations — including an AML interpreter, a namespace manager, ACPI event and resource management, and ACPI hardware support.
%
%One of the goals of the ACPICA Subsystem is to provide an abstraction level high enough such
%that the host operating system does not need to understand or know about the very low-level ACPI
%details. For example, all AML code is hidden from the host. Also, the details of the ACPI hardware
%are abstracted to higher-level software interfaces.
%
%The ACPICA Subsystem implementation makes no assumptions about the host operating system or environment. The only way it can request operating system services is via interfaces provided by the OS Services Layer.
%
%The primary user of the services provided by the ACPICA Subsystem are the host OS device drivers and power/thermal management software.
%
\subsubsection{Operating System Services Layer \gls{osl}}
\say{OS Services Layer (OSL)} act as a request translation service for host os from OS-independent ACPICA subsystem. The OSL develops a common subset for interfaces of OS service by utilizing the primitives usable from host OS.

The OSL has to be developed afresh for each and every supported host OS. There exists only one ACPICA Subsystem which OS-independent but there has to be a different OSL for each OS backed by the ACPICA.

The whole ACPICA in relation to the host OS is portrayed in Figure \ref{fig:introduction-acpica-subsystem-architecture}

\begin{figure}[!htbp]
	\centering
	\includegraphics[width=0.7\linewidth]{introduction/acpica-subsystem-architecture}
	\caption{ACPICA Subsystem Architecture}\label{fig:introduction-acpica-subsystem-architecture}
\end{figure}

\subsubsection{\gls{acpica} Subsystem Interaction}
ACPICA Subsystem develops a subset of external interface links that could directly summoned via host OS. These Acpi interfaces serve the literal ACPI services for host. When OS services are needed while servicing of request of an ACPI the Subsystem makes oblique request to host OS through the fixed AcpiOs interfaces. 

\begin{figure}[!htbp]
  \centering
  \includegraphics[width=0.7\linewidth]{introduction/acpi-interaction-between-the-architectural-components}
  \caption{Interaction between the Architectural Components}\label{fig:-introduction-acpi-interaction-between-the-architectural-components}
\end{figure}

Figure \ref{fig:-introduction-acpi-interaction-between-the-architectural-components} portrays the kinship and fundamental interaction linking the diverse architectural modules by screening the control flow among them. Note that OS independent ACPICA Subsystem could never call the host OS directly and instead it has to make call(s) to the AcpiOs interfaces inside the OSL. This serves the ACPICA code as OS-independence.

\chapter{Design and Architecture}\label{chapter-design-and-architecture}
\lhead{Chapter 2. \emph{Design and Architecture}}
\section{Design}\label{section-design}

\subsection{Design Overview of \gls{uefi}}
The design of UEFI is based on the following fundamental elements:

\begin{itemize}
	\item \textbf{Reuse of existing table-based interfaces} - In order to preserve investment in existing infrastructure support code, both within the OS and firmware, variety of existing specifications that are commonly implemented on platforms compatible with supported processor specifications must be implemented on platforms wishing to comply with the UEFI specification.
	\item \textbf{System partition} defines a partition and file system that are designed to allow safe sharing between multiple vendors, and for different purposes. The ability to include a separate, shareable system partition presents an opportunity to increase platform value-add without significantly growing the need for nonvolatile platform memory
	\item \textbf{Boot services} provide interfaces for devices and system functionality that
	can be used during boot time. Device access is abstracted through "handles" and
	"protocols". This facilitates reuse of investment in existing BIOS code by keeping
	underlying implementation requirements out of the specification without burdening the
	consumer accessing the device.
	\item \textbf{Runtime services} - A minimal set of runtime services is presented to ensure appropriate	abstraction of base platform hardware resources that may be needed by the OS during its	normal operations.
\end{itemize}

\begin{figure}[h]
	\centering
	\includegraphics[width=0.8\linewidth]{design/uefi-conceptual-overview}
	\caption{UEFI Conceptual Overview}\label{fig:design-uefi-conceptual-overview}
\end{figure}

\textbf{Error! Reference source not found} illustrates the interactions of the various components of an UEFI specification-compliant system that are used to accomplish platform and OS boot.

The platform firmware can retrieve the OS loader image from the System Partition. The
specification provides for a variety of mass storage device types including disk, CD-ROM, and
DVD as well as remote boot via a network. Through the extensible protocol interfaces, it is possible
to add other boot media types, although these may require OS loader modifications if they require
use of protocols other than those defined in this document.

Once started, the OS loader continues to boot the complete operating system. To do so, it
may use the EFI boot services and interfaces defined by this or other required specifications to
survey, comprehend, and initialize the various platform components and the OS software that
manages them. EFI runtime services are also available to the OS loader during the boot phase.

\subsubsection{UEFI Driver Goals}
The UEFI Driver Model has the following goals:
\begin{itemize}
	\item \textbf{Compatible} - Drivers conforming to this specification must maintain compatibility with the EFI and the UEFI Specification. This means that the UEFI Driver Model takes advantage of the extensibility mechanisms in the UEFI Specification to add the required
	functionality.
	\item \textbf{Simple} - Drivers that conform to this specification must be simple to implement and	simple to maintain. The UEFI Driver Model must allow a driver writer to concentrate on
	the specific device for which the driver is being developed. A driver should not be
	concerned with platform policy or platform management issues. These considerations
	should be left to the system firmware.
	\item \textbf{Scalable} - The UEFI Driver Model must be able to adapt to all types of platforms. These platforms include embedded systems, mobile, and desktop systems, as well as workstations and servers.
	\item \textbf{Flexible} - The UEFI Driver Model must support the ability to enumerate all the devices, or to enumerate only those devices required to boot the required OS. The minimum device
	enumeration provides support for more rapid boot capability, and the full device enumeration provides the ability to perform OS installations, system maintenance, or system diagnostics on any boot device present in the system.
	\item \textbf{Extensible} - The UEFI Driver Model must be able to extend to future bus types as they
	are defined.

	\item \textbf{Portable} - Drivers written to the UEFI Driver Model must be portable between platforms and between supported processor architectures.
	\item \textbf{Interoperable} - Drivers must coexist with other drivers and system firmware and must do so without generating resource conflicts.
	\item \textbf{Describe complex bus hierarchies} - The UEFI Driver Model must be able to describe a variety of bus topologies from very simple single bus platforms to very complex platforms
	containing many buses of various types.

	\item \textbf{Small driver footprint} - The size of executables produced by the UEFI Driver Model must be minimized to reduce the overall platform cost. While flexibility and extensibility
	are goals, the additional overhead required to support these must be kept to a minimum to
	prevent the size of firmware components from becoming unmanageable.
	\item \textbf{Address legacy option rom issues} - The UEFI Driver Model must directly address and	solve the constraints and limitations of legacy option ROMs. Specifically, it must be
	possible to build add-in cards that support both UEFI drivers and legacy option ROMs,
	where such cards can execute in both legacy BIOS systems and UEFI-conforming latforms, without modifications to the code carried on the card. The solution must provide an evolutionary path to migrate from legacy option ROMs driver to UEFI drivers.
\end{itemize}

\subsection{\gls{uefi}/\gls{pi} Firmware Images}
\gls{uefi} and \gls{pi} specifications define the standardized format for EFI firmware storage devices (FLASH or other non-volatile storage) which are abstracted into "Firmware Volumes". Build systems must be capable of processing files to create the file formats described by the \gls{uefi} and PI specifications. The tools provided as part of the \gls{edk2} BaseTools package process files compiled by third party tools, as well as text and Unicode files in order to create UEFI or PI compliant binary image files. In some instances, where UEFI or PI specifications do not have an applicable input file format, such as the Visual Forms Representation (VFR) files used to create PI compliant IFR content, tools and documentation have been provided that allows the user to write text files that are processed into formats specified by UEFI or PI specifications.

\begin{figure}[h]
	\centering
	\includegraphics[width=0.8\linewidth]{design/uefi-pi-firmware-image-creation}
	\caption{UEFI/PI Firmware Image Creation}\label{fig:design-uefi-pi-firmware-image-creation}
\end{figure}

A Firmware Volume (FV) is a file level interface to firmware storage. Multiple FVs may be present in a single FLASH device, or a single FV may span multiple FLASH devices. An FV may be produced to support some other type of storage entirely, such as a disk partition or network device. For more information consult the Platform Initialization Specification, Volume 3.
In all cases, an FV is formatted with a binary file system. The file system used is typically the Firmware File System (FFS), but other file systems may be possible in some cases. Hence, all modules are stored as "files" in the FV. Some modules may be "execute in place" (linked at a fixed address and executed from the ROM), while others are relocated when they are loaded into memory and some modules may be able to run from ROM if memory is not present (at the time of the module load) or run from memory if it is available.
Files themselves have an internally defined binary format. This format allows for implementation of security, compression, signing, etc. Within this format, there are one or more "leaf" images. A leaf image could be, for example, a PE32 image for a DXE driver.

Therefore, there are several layers of organization to a full UEFI/PI firmware image. These layers are illustrated below in Figure \ref{fig:design-uefi-pi-firmware-image-creation}. Each transition between layers implies a processing step that transforms or combines previously processed files into the next higher level. Also shown in Figure \ref{fig:design-uefi-pi-firmware-image-creation} are the reference implementation tools that process the files to move them between the different layers.

\begin{figure}[h]
	\centering
	\includegraphics[width=0.9\linewidth]{design/efi-application-creation}
	\caption{UEFI/PI Firmware Image Creation}\label{fig:design-efi-application-creation}
\end{figure}


In addition to creating images that initialize a complete platform, the build process also supports creation of stand-alone UEFI applications (including OS Loaders) and Option ROM images containing driver code. Figure \ref{fig:design-efi-application-creation}, below, shows the reference implementation tools and creation processes for both of these image types

The final feature that is supported by the EDK II build process is the creation of Binary Modules that can be packaged and distributed for use by other organizations. Binary modules do not require distribution of the source code. This will permit vendors to distribute UEFI images without having to release proprietary source code.

This packaging process permits creation of an archive file containing one or more binary files that are either Firmware Image files or higher (EFI Section files, Firmware File system files, etc.). The build process will permit inserting these binary files into the appropriate level in the build stages.

\subsection{Platform Initialization \gls{pi} Boot Sequence}
Platform Initialization \gls{pi} compliant system firmware has to support the six phases: 
\begin{enumerate}
	\item Security (\gls{sec}) Phase
	\item Pre-efi Initialization (\gls{pei}) Phase
	\item Driver Execution Environment (\gls{dxe}) Phase
	\item Boot device selection (\gls{bds}) Phase
	\item Run time (RT) services and After Life (AL) (transition from the OS back to the firmware) of system. 
\end{enumerate}
Figure \ref{fig:design-pi-boot-phases} describes the phases and transition in detail.

\begin{figure}[h]
	\includegraphics[width=\linewidth]{PI_Boot_Phases}
	\caption{\gls{pi} Boot Phases}\label{fig:design-pi-boot-phases}
\end{figure}

\subsection{Security (\gls{sec})}
The Security (SEC) phase is the initial phase in the PI Architecture and is liable for the following:
\begin{itemize}
	\item Handling restart events of all platform
	\item Creation of a temporary memory store
	\item Bringing the root of trust in the system
	\item Transit handoff information to next phase - the PEI Foundation
\end{itemize}
The security section may have the modules with source code scripted in assembly language. Hence, some \gls{edk2} module development environment (MDE) modules can consist of assembly code. During Occurrence of this, both Windows and GCC versions of assembly language code are served in different files.

\subsection{Pre-EFI Initialization (\gls{pei})}
The Pre-EFI Initialization (PEI) phase described in the PI Architecture specifications is invoked quite betimes in the boot period. Specifically, after about preliminary processing in the Security (SEC) phase, any machine restart event will invoke the PEI phase.
The PEI phase is designed to be developed in many parts and consists of:
\begin{itemize}
	\item PEI Foundation (core code)
	\item Pre-EFI Initialization Modules (specialized plug-ins)
\end{itemize}
The PEI phase at first operates with the platform in a developing state, holding only on-processor resources, such as the cache of processor for call stack, to dispatch the Pre-EFI Initialization Modules (PEIMs).

The PEI phase cannot assume the availability of amounts of memory (RAM) as DXE and hence PEI phase limits its support to the following:
\begin{itemize}
	\item Locating and validating PEIMs
	\item Dispatching PEIMs
	\item Facilitating communication between PEIMs
	\item Providing handoff data to later phases
\end{itemize}

These PEIMs are responsible for the following:
\begin{itemize}
	\item Initializing some permanent memory complement
	\item Characterizing the memory in Hand-Off Blocks (HOBs)
	\item Characterizing the firmware volume locations in HOBs
	\item Transit the control into next phase - the Driver Execution Environment (DXE) phase
\end{itemize}

Figure \ref{fig:design-pei-operation-diagram} shows a diagram describes the action carried out during the PEI phase

\begin{figure}[h]
	\centering
	\includegraphics[width=0.8\linewidth]{design/pei-operation-diagram}
	\caption{Diagram of PI Operations}\label{fig:design-pei-operation-diagram}
\end{figure}

\subsubsection{PEI Services}
The PEI Foundation institutes a system table for the PEI Services named as PEI Services Table that is viewable to all \gls{pei} Modules (PEIMs) in the system. A PEI Service is defined as a method, command or other potentiality manifested by the PEI Foundation when that service's initialization needs are met. As the PEI phase having no permanent memory available until almost the end of the phase, all the various types of services created during this phase (PEI phase) cannot be as enrich as those created during later phases. A pointer to PEI Services Table is sent into entry point of each PEIM's and also to part of each PEIM-to-PEIM Interface (PPI) because the location of PEI Foundation and its temporary memory is unknown at build time. 

The PEI Foundation provides the classes of services listed in Table \ref{table:design-pei-foundation-class-service}

\begin{table}[h]
	\centering
	\renewcommand*{\arraystretch}{2}
	\caption{Services provided by PEI Foundation Classes}\label{table:design-pei-foundation-class-service}
	\begin{tabular}{ l | p{9cm} }
		Service & Details
		\\ \hline \hline
		PPI Services & Manages PPIs to ease inter-module method calls between PEIMs. A database maintained in temporary RAM to track installed interfaces.
		\\ \hline
		Boot Mode Services & Manages the boot mode (S3, S5, diagnostics, normal boot, etc.)
		\\ \hline
		HOB Services & Creates data structures (Hand-off-blocks) that are used to convey information to the next phase 
		\\ \hline
		Firmware Volume Services & Finds PEIMs and along with that other firmware files in the firmware volumes
		\\ \hline
		PEI Memory Services & provides a collection of memory management services (to be used before and after permanent memory to discovered)
		\\ \hline
		Status Code Services & Provides general progress and error code reporting services (i.e. port 080h or a serial port for text output for debug)
		\\ \hline
		Reset Services & Provides a common means to aid initializing warm or cold restart of the system
		\\ \hline
	\end{tabular}

\end{table}


\subsubsection{PEI Foundation}
The PEI Foundation is the entity that carried outs following activity:
\begin{itemize}
	\item Dispatching of Pre-EFI initialization modules (PEIMs)
	\item Maintaining the boot mode
	\item Initialization of permanent memory
	\item Invoking the DXE loader 
\end{itemize}
The PEI Foundation written to be portable across all the various platforms architecture of a given instruction-set. i.e. A binary for IA-32 (32-bit Intel architecture) works across all Pentium processors and similarly Itanium processor family work across all Itanium processors.

Irrespective of the processor micro architecture, the set of services uncovered by the PEI Foundation should be the same. This consistent surface area around the PEI Foundation allows PEIMs to be written in the \verb|C programming language| and compiled across any micro architecture.

\subsection{PEI Dispatcher}
The PEI Dispatcher is basically a state machine which is implemented in the PEI Foundation. The PEI Dispatcher evaluates the dependency expressions in Pre-EFI initialization modules (PEIMs) that are lying in the \gls{fv}s being examined.

Dependency expressions are coherent combinations of PEIM-to-PEIM Interfaces (PPIs). These expressions distinguish the PPIs that must be available for use before a given PEIM can be invoked. The PEI Dispatcher references the PPI database in the PEI Foundation to conclude which PPIs have to be installed and evaluate the dependency expression for the PEIM. If PPI has already been installed then dependency expression will evaluate to \verb|TRUE|, which notifies  PEI Dispatcher it can run PEIM. At this stage, the PEI Foundation handovers control to the PEIM with \verb|TRUE| dependency expression. 

The PEI Dispatcher will exit Once the PEI Dispatcher has examined and evaluated all of the PEIMs in all of the uncovered firmware volumes and no more PEIMs can be dispatched (i.e. the dependency expressions (\gls{depex}) do not evaluate from \verb|FALSE| to \verb|TRUE|). At this stage, the PEI Dispatcher cannot invoke any additional PEIMs. The PEI Foundation then takes back control from the PEI Dispatcher and calls the \verb|DXE IPL PPI| to navigate control to the DXE phase of execution.

\subsection{Drive Execution Environment (\gls{dxe})}
Before the DXE phase, the Pre-EFI Initialization (PEI) phase is held responsible for initializing permanent memory in the platform. Hence, DXE phase can be loaded and executed. At the very end of the PEI phase, state of the system is handed over to the DXE phase through Hand-Off Blocks (list of position independent data structures). 

There are three components in the DXE phase:
\begin{enumerate}
	\item DXE Foundation
	\item DXE Dispatcher
	\item A set of DXE Drivers
\end{enumerate}

\subsection{Boot Device Selection (\gls{bds})}
The BDS Architectural Protocol has part of implementation of the Boot Device Selection (BDS) phase. After evaluation of all of the DXE drivers dependencies, DXE drivers with satisfied dependencies are loaded and executed by the DXE Dispatcher, the DXE Foundation will pass the control to the BDS Architectural Protocol. The BDS phase held responsible for the following:

\begin{itemize}
	\item Initializing console devices
	\item Loading device drivers
	\item Attempt of loading and executing boot selections
\end{itemize}

\subsection{Transient System Load (TSL) and Runtime (RT)}
Primarily the OS vendor provides boot loader known as The Transient System Load (TSL). TSL and Runtime Services (RT) phases may allow access to persistent content, via UEFI drivers and applications. Drivers in this category include PCI Option ROMs.

\subsection{After Life (AL)}
The After Life (AL) phase contains the persistent UEFI drivers used to store the state of the system during the OS systematically shutdown, sleep, hibernate or restart processes.



\subsection{Generic Build Process}
All code starts out as either C sources and header files, assembly sources and header files, UCS-2 HII strings in Unicode files, Virtual Forms Representation files or binary data (native instructions, such as microcode) files. Per the UEFI and PI specifications, the C and Assembly files must be compiled and linked into PE32/PE32+ images.
While some code is designed to execute only from ROM, most UEFI/PI modules are written to be relocate-able. These are written and built different. For example, Execute In Place (XIP) module code is written and compiled to run from ROM, while the majority of the code is written and compiled to execute from memory, which requires that the code be relocate able.
Some modules may also permit dual mode, where it will execute from memory only if memory is available, otherwise it will execute from ROM. Additionally, modules may permit dual access, such as a driver that contains both PEI and DXE implementation code. Code is assembled or compiled, then linked into PE32/PE32+ images, the relocation section may or may not be stripped and an appropriate header will replace the PE32/PE32+ header. Additional processing may remove more non-essential information, generating a Terse (TE) image.
The binary executables are converted into EFI firmware file sections. Each module is converted into an EFI Section consisting of an Section header followed by the section data (driver binary).

\subsubsection{EFI Section Files}
he general section format for sections less than 16MB in size is shown in Figure \ref{fig:design-general-efi-section-format}. Figure \ref{fig:design-general-efi-section-format-large} shows the section format for sections 16MB or larger in size using the extended length field.

\begin{figure}[h]
	\centering
	\includegraphics[width=0.7\linewidth]{design/general-efi-section-format-large}
	\caption{General EFI Section Format for large size Sections(greater then 16 MB)}\label{fig:design-general-efi-section-format-large}
\end{figure}


\begin{figure}[h]
	\centering
	\includegraphics[width=0.7\linewidth]{design/general-efi-section-format}
	\caption{General EFI Section Format (less then 16 MB)}\label{fig:design-general-efi-section-format}
\end{figure}



\subsection{Cross Compatibility of CPUs}
Whenever customer try to change the default Intel motherboard CPU with different Intel silicon chip which won’t works. The specific CPU Chip initialization varies for each generation. So, the board designs should be designed is such a way that specific generation CPU should support., if we change the CPU with a different Intel Board it will not even boot, because the BIOS doesn't support for other Silicon Initialization for other CPUs.

So, we are integrating the runtime detection of the silicon during the Pre-Extensible Firmware Initialization (\gls{pei}) phase. So, within single Integrated Firmware Image (\gls{ifwi}) should support the Multi Generation CPUs which is never tried before.

Each silicon has a fixed register from which the CPU generation can be identified., so the BIOS should read that register and program in such a way the is CPU1 is in Platform it should support the CPU1 Features like PCIe, Graphics \& DMI., if the CPU1 is replaced with CPU2 then it should support the CPU2 speed. That should be taken care by the BIOS.

\begin{figure}[h]
	\centering
	\includegraphics[width=0.7\linewidth]{design/cross-compatibility-design}
	\caption{Cross Compatibility Design}\label{fig:design-cross-compatibility-design}
\end{figure}

Figure \ref{fig:design-cross-compatibility-design} shows the general view of the Cross Compatibility of CPUs.

BIOS is the part of Integrated Firmware Image which resides at the End of the Binary table. The CPU swap can only occur in Specially designed Intel Designed Board only. Mainly because for each and every feature it required some hardware(H/W) requirements. If that H/W requirement not present. Then It will boot but doesn't support the Maximum speed.


\begin{figure}[h]
	\centering
	\includegraphics[width=0.7\linewidth]{design/bios-support-for-cross-compatibility}
	\caption{BIOS Support for Cross Compatibility}\label{fig:design-bios-support-for-cross-compatibility}
\end{figure}

Figure \ref{fig:design-bios-support-for-cross-compatibility} shows the BIOS role for identifying the CPUs during PEI phase.

As the number of Feature increases in the Silicon BIOS size also increases, usually the BIOS size varies from Platform to Platform and CPU to CPU., as we are integrating the Compatibility the BIOS size obviously increases. 

The structure of \gls{ifwi} is Shown in Figure \ref{fig:design-integrated-firmware-image}

\begin{figure}[h]
	\centering
	\includegraphics[width=0.7\linewidth]{design/integrated-firmware-image}
	\caption{Integrated Firmware Image}\label{fig:design-integrated-firmware-image}
\end{figure}



\section{Architecture of BIOS Firmware}\label{section-architecture}
\subsection{Overview}
If you interpret BIOS image as close look then it is nothing but the file system which is made in a byte format to be read by low level programming language which is most efficient method to store the data or content.

The concept of initialization of Platform includes the execution of this BIOS image which is stored on the every \gls{soc}

The components which plays role in Platform Initialization are listed below:
\begin{itemize}
	\item Firmware Volume (\gls{fv}) - consists of one or more firmware file systems
	\item Firmware File System (\gls{ffs}) which consists of one or more firmware files
	\item Firmware File - Encapsulated section or leaf section
	\item Reference Layout of Binary
	\item Pre-EFI Initialization (\gls{pei}) PEIM to PEIM Interfaces (PPIs)
	\item Driver Execution Environment (\gls{dxe}) Protocols
\end{itemize}

\subsection{Design of Firmware Storage}
Design of firmware storage elaborates the way that how files needs to be stored and accessed in nonvolatile storage environment. Implementation of any firmware has to support and follow the standard structure for \gls{pi} Firmware Volume and the structure of \gls{ffs}.

\paragraph{Firmware Device} - a persistent physical repository consisting data and/or firmware code. Typically it is a component of flash but may also be any other type of persistent storage. Singular physical firmware device can be partitioned in to multiple other smaller pieces to form many other logical firmware devices from it and vice-versa.


\paragraph{Flash} devices are most usual nonvolatile storage mechanism for firmware volumes. Often, flash devices are partitioned into many sectors or blocks of potentially differing sizes, each along with various runtime characteristics.

In the design of Firmware File System (\gls{ffs}), several observed unique qualities of flash devices are listed below:
\begin{itemize}
	\item Erase operation processed on the basis of sector-by-sector. After ensuring, every bits of sector return their \verb|erase value|\footnote{either all $0$ or all $1$}.
	\item Write operation can be performed on a bit-by-bit basis. i.e. In case erase value is $ 0 $ then bit value $ 0 $ can be changed to $ 1 $.
	\item Only by performing erase operation on the whole flash sector, \verb|non-erase value| can change to \verb|erase value|.
	\item Capable of enable/disable reads and writes to individual flash sectors or the entire flash.
	\item Operations like writes and erases are much longer than reads operation.
	\item Many times places restraints on the trading operations that can be executed while a write or erase is in progress.
\end{itemize}

\subsection{Firmware Volume (\gls{fv})}
The BIOS image is consisting of one or more logical firmware devices known as a Firmware Volume (\gls{fv}). Firmware Volume is the very basic and efficient logical storage mechanism for data and/or code. If you consider file system as a basic unit then firmware volume is unionized into these one or more file system units.
Table \ref{table:firmware-volume-attributes} describes attributes in each firmware volume.

\begin{table}[!htbp]
	\centering
	\renewcommand{\arraystretch}{2}
	\caption{Firmware Volume Attributes}\label{table:firmware-volume-attributes}
	\begin{tabular}{p{4cm} | p{11cm}}
		\textbf{Attribute} & \textbf{Description}
		\\ \hline \hline
		Name & each volume has a unique identifier name having UEFI Globally Unique Identifier (GUID). 
		\\ \hline
		Size & describes total size of all data (includes all information like headers, files and free/reserved space)
		\\ \hline
		Format & describes type of Firmware File System (FFS) which is unionized in construction of the volume.
		\\ \hline
		Memory is Mapped or not? & some volumes may requires to be memory-mapped which determines whether the entire content of the volume can appear at once in the processor's memory address space. 
		\\ \hline
		Sticky Write? & Specifies whether or not special erase cycles requires in order to change value of bits into an erase value from non-erase value
		\\ \hline
		Erase Polarity & In case a volume supports \textit{Sticky Write}, then after processing an erase cycle every bits in the volume will return to this value ($ 0 $ or $ 1 $)
		\\ \hline
		Alignment & A volume is required to be aligned on some power-of-two ($ 2^x $) boundary such that $ minimum >= \text{highest file alignment value} $.
		\\ \hline
		Enable/Disable Read capable status & Decides whether to keep volumes as hidden from readable or not
		\\ \hline
		Enable/Disable Write capable Status & Decides whether to keep volumes as hidden from writable or not
		\\ \hline
		Lock Capable/Status & Volumes could also have their locking mechanism
		\\ \hline
		Read-Lock Capable/Status & Volumes could also have the power to lock their read status
		\\ \hline
		Write-Lock Capable/Status & Volumes could also have the power to lock their write status
		\\ \hline
	\end{tabular}
\end{table}

Apart from this Firmware volumes also consisting of few more information about the correspondence between OEM file types and a \verb|GUID|.

\subsection{Firmware File System (\gls{ffs})}
The logical data payload within firmware volume is known as a Firmware File System (\gls{ffs}) which illustrates the structure of files and free space (if any). To affiliate a driver to firmware volume every firmware file systems contains a globally unique ID (GUID).


\begin{table}[h]
	\centering
	\renewcommand{\arraystretch}{2}
	\caption{Firmware Files Attributes}\label{table:firmware-files-attributes}
	\begin{tabular}{p{4cm} | p{11cm}}
		\textbf{Attribute} & \textbf{Description}
		\\ \hline \hline
		Name & each volume has a unique identifier name having UEFI Globally Unique Identifier (\gls{guid}). Name of the File(s) has to be unique within a same firmware volume.
		\\ \hline
		Type & Type of the individual file which can be Normal, OEM, Debug, FV Specific. More file types information are described in Figure \ref{fig:firmware-file-types}.
		\\ \hline
		Alignment & Every data of file to be aligned on some power-of-two ($ 2^x $) boundary such that these boundaries are founded depending on the alignment of firmware volume.
		\\ \hline
		Size & Describes size of each file which consists of data of size zero or more bytes
		\\ \hline
	\end{tabular}
\end{table}

Firmware files consists of code or raw data or both. Attributes of files are described in Table \ref{table:firmware-files-attributes}.

Integrity check and staged file creation are some extra attribute formats which might spotted in some firmware volume. Firmware File Sections are unit which unionized in a standard fashion to form certain file types for the file data.

OEM file types (described in detail in Figure \ref{fig:firmware-file-types}) enables to support non-standard file types.

PEI phase is responsible to serve the file related services which are carried out using PEI Service Table. On the Other hand the \\ \verb|EFI_FIRMWARE_VOLUME2_PROTOCOL| services which are attached to a volume's handle (\verb|ReadFile|, \verb|ReadSection|, \verb|WriteFile| and \verb|GetNextFile|) are responsible to carried out file related services in DXE phase.

\subsubsection{Firmware File Types}
If you consider an application with file name such as \texttt{XYZ.exe}, in which content format of \verb|XYZ.exe| is implied by the ".exe" in the file name. Based on the situation of operation, this extension normally signals the contents of \verb|XYZ.exe|. The PI Firmware File System characterizes the contents of a file that is returned by the firmware volume interface.

Firmware File System of the Platform Initialization dictates an enumeration of many file types. For example, the type
\verb|EFI_FV_FILETYPE_RAW| implies that the file is a RAW Binary Data. In the same way, files with the type \verb|EFI_FV_FILETYPE_SMM_CORE| supports MM traditional mode .

\begin{figure}[!htbp]
	\centering
	\includegraphics[width=0.9\linewidth]{architecture/firmware-file-type}
	\caption{Firmware File Type}\label{fig:firmware-file-types}
\end{figure}

\subsection{Firmware File Section}
Firmware File Section is individual distinct unit of certain file types which has following attributes:
\begin{table}[!htbp]
	\begin{tabular}{l | p{9cm}}
		Attribute & Description
		\\ \hline \hline
		Type & Each section has type 
		\\ \hline
		Size & describes size of the section
		\\ \hline
	\end{tabular}
\end{table}

However as many as types of sections are present, they eventually fall in one of the below broadly described categories:
\begin{itemize}
	\item \textbf{Encapsulation section} - logical storage consisting of the one or more section.
	The child section(s) which are lying within the encapsulation section (parent section) can be another encapsulation section or a leaf section which are also called relative peers to each other. An encapsulation section never consists of data in itself; however it is just a container that ultimately ends in leaf section(s). Files which are stacked with section can be imagined as tree consisting of nodes (encapsulation section) and	leaves (leaf section). The root which can be interpreted as the file image itself may have a discrete number of sections. Sections that exist in the root have no parent section but are still considered peers.
	
	\item \textbf{Leaf Sections} - Contrary to the encapsulation section, leaf section does contain data and only data within it. Type of section defines which kind of data is stored within the leaf section.
\end{itemize}

\begin{figure}[!htbp]
	\centering
	\includegraphics[width=0.9\linewidth]{architecture/firmware-file-system-representation}
	\caption{Example File System Image}\label{fig:architecture-firmware-file-system-representation}
\end{figure}

As illustrated in Figure \ref{fig:architecture-firmware-file-system-representation}, the root which we interpret as the file image has two encapsulation sections which are $ E0 $, $ E1 $ and one leaf section which is $ L3 $. $ E0 $ which is the first encapsulation section possessing three child node which are all leaves ($ L0 $, $ L1 $, and $ L2 $). $ E1 $ is the another encapsulation section which possesses only two children, where one of them is encapsulation ($ E2 $) and the another is the leaf ($ L6 $). $ E2 $ which is the very last encapsulation section consists two children which are both leaves only ($ L4 $ and $ L5 $).

With the help of \verb|FfsFindSectionData|, services related to section are populated with the help of PEI Service Table in the PEI phase. On the other hand \verb|ReadSection| which is attached to service protocol \verb|EFI_FIRMWARE_VOLUME2_PROTOCOL| responsible to populate services related to section during the DXE phase.

\subsection{Firmware File Section Types}
Subjective types of section are described in Table \ref{table:architectural-section-types}.

\begin{table}[!htbp]
	\centering
	\renewcommand{\arraystretch}{1.2}
	\caption{Types of Section}\label{table:architectural-section-types}
	\begin{tabular}{p{7cm} | l | p{5cm}}
		Name of Section & Value & Details
		\\ \hline \hline
		\verb|EFI_SECTION_COMPRESSION| & $ 0x1 $ & non-leaf section containing compressed section(s) within
		\\ \hline
		\verb|EFI_SECTION_GUID_DEFINED| & $ 0x2 $ & non-leaf section which only to be used while in process of build and not for execution
		\\ \hline
		\verb|EFI_SECTION_DISPOSABLE| & $ 0x3 $ & non-leaf section which only to be used while in process of build and not for execution
		\\ \hline
		\verb|EFI_SECTION_PE32| & $ 0x10 $ & Image executable of PE32+
		\\ \hline
		\verb|EFI_SECTION_PIC| & $ 0x11 $ & Code independent of position
		\\ \hline
		\verb|EFI_SECTION_TE| & $ 0x12 $ & Image of Terse Executable
		\\ \hline
		\verb|EFI_SECTION_DXE_DEPEX| & $ 0x13 $ & Expression for DXE driver dependency
		\\ \hline
		\verb|EFI_SECTION_VERSION| & $ 0x14 $ & version of the section - text/numeric
		\\ \hline
		\verb|EFI_SECTION_USER_INTERFACE| & $ 0x15 $ & Human readable and easily interpretable name for driver
		\\ \hline
		\verb|EFI_SECTION_COMPATIBILITY16| & $ 0x16 $ & 16-bit exe of DOS fashion
		\\ \hline
		\verb|EFI_SECTION_FIRMWARE_VOLUME_IMAGE| & $ 0x17 $ & PI Firmware Volume Image
		\\ \hline
		\verb|EFI_SECTION_FREEFORM_SUBTYPE_GUID| & $ 0x18 $ & Raw data with GUID in header to define format
		\\ \hline
		\verb|EFI_SECTION_RAW| & $ 0x19 $ & Raw data
		\\ \hline
		\verb|EFI_SECTION_PEI_DEPEX| & $ 0x1b $ & Expression for PEI driver dependency
		\\ \hline
		\verb|EFI_SECTION_SMM_DEPEX| & $ 0x1c $ & Leaf section which determine the order of dispatch for the MM Traditional driver in SMM.
		\\ \hline
	\end{tabular}
\end{table}

\subsection{PI Architecture Firmware File System Format}
Basic encoding of binary used for PI firmware file, firmware volume and file system is illustrated in this section. Development which carries out the non-vendor firmware files or firmware volumes to be enclosed into the system must have the standard formats. These sections also describes the way features of the standard format mapped into the standard interfaces of DXE and PEI.

The standard format of firmware file and volume also brings in extra dimensions and potential that are used to assure the unity of firmware volume. The standard format is unionized by three different levels: firmware volume, firmware file system, and firmware file.

\begin{figure}[!htbp]
	\centering
	\includegraphics[width=0.8\linewidth]{architecture/the-firmware-volume-format}
	\caption{The Firmware Volume Format}\label{fig:architecture-the-firmware-volume-format}
\end{figure}

The guided formatting of firmware volume (Figure \ref{fig:architecture-the-firmware-volume-format}) made of two parts: 
The FV header and FV data. Header of FV describes every attributes mentioned in “Firmware Volumes” in Table \ref{table:firmware-volume-attributes}. This header also has \gls{guid} which identifies format of the firmware file system utilized to orchestrate data in the firmware volume. The \say{firmware volume header} is compatible with every other firmware file systems except the PI Firmware File System.

\say{Firmware File System format} explains the way the firmware files and free space are conceived inside the firmware volume. However on the other hand \say{Firmware File format} describes how files are organized. The firmware file format made of two parts: the firmware file header and the firmware file data.

\subsubsection{Firmware Volume Format}
The PI Architecture Firmware Volume format key outs the binary structure of a firmware volume. The firmware volume format possesses a FV header followed by the FV data. The FV header is represented by variable \verb|EFI_FIRMWARE_VOLUME_HEADER|.
The format layout of the FV data is described by a GUID. Valid files system GUID values are \verb|EFI_FIRMWARE_FILE_SYSTEM2_GUID| and \verb|EFI_FIRMWARE_FILE_SYSTEM3_GUID|.

\subsubsection{Firmware File System Format}
The PI Architecture Firmware File System is a binary design of logical file storage within firmware volumes. It is a flat file system in which there is no rendering of any directory hierarchy structure. Each and every files lies directly in the root of the storage. Files are stored end to end without any directory entry to explain which files are present. Parsing the information stored in a firmware volume to find a itemization of files exists needs the complete walk through over the firmware volume in and out.

\paragraph{Firmware File System GUID} The firmware volume header has a unique data field for the file system GUID. The two valid FFS file systems are defined by the GUID values in variable \verb|EFI_FIRMWARE_FILE_SYSTEM2_GUID| and \verb|EFI_FIRMWARE_FILE_SYSTEM3_GUID|. In case of the FFS file system, if it does allows files larger than $ 16 \ MB $ along with backward compatibility \verb|EFI_FIRMWARE_FILE_SYSTEM2_GUID| then \verb|EFI_FIRMWARE_FILE_SYSTEM3_GUID| is used.

\paragraph{Volume Top File} known as VTF is a file that has to be presented such that the very last byte of file is also the very last byte of the firmware volume. Irrespective of type of the file, a VTF have to have GUID for the file name which is declared as variable \verb|EFI_FFS_VOLUME_TOP_FILE_GUID|.
Driver cide if Firmware file system has to be exposed of this GUID and infix an alignment pad file as and when needed to assure that the VTF is situated correctly at the top of the firmware volume. Length and alignment of File requirements needs to be coherent with the top of volume so that a write error does occurs and the unwanted firmware volume modification can be prevented.

\subsection{Firmware File Format (\gls{ffs})}
Every FFS files begins with its header data that is aligned on an $ 8-byte $ (which is power-of-two $ 2^3 $) boundary with respect to the origin of the firmware volume. FFS files consists of the below parts:
\begin{itemize}
	\item Header
	\item Data
\end{itemize}

When a zero-length file is created without any data it still has to have header and will consume minimum $ 24 bytes $ of space.

The data (if any) exists in file then it immediately conjugated after the header. How the data within a file
is formed can be identified by the \say{Type} field in the header which can be either of \verb|EFI_FFS_FILE_HEADER| and \verb|EFI_FFS_FILE_HEADER2|.

Figure \ref{fig:architecture-typical-ffs-file-layout-less-than-16MB} exemplifies the typical layout of a (i.e. \verb|EFI_FFS_FILE_HEADER|) \say{PI Architecture Firmware File} $ (\leq 16 Mb) $.

\begin{figure}[!htbp]
	\centering
	\includegraphics[width=0.8\linewidth]{architecture/typical-ffs-file-layout-less-than-16MB}
	\caption{Layout representation of FFS File Header $ (\leq 16 Mb) $ }\label{fig:architecture-typical-ffs-file-layout-less-than-16MB}
\end{figure}

Figure \ref{fig:architecture-typical-ffs-file-layout-greater-than-16MB} exemplifies the typical layout of \say{PI Architecture Firmware File} $ (> 16 Mb) $.

\begin{figure}[!htbp]
	\centering
	\includegraphics[width=0.8\linewidth]{architecture/typical-ffs-file-layout-greater-than-16MB}
	\caption{Layout representation of FFS File Header 2 layout for files $ (> 16Mb) $}\label{fig:architecture-typical-ffs-file-layout-greater-than-16MB}
\end{figure}


\subsection{Firmware File Section Format}
Storage Data format mechanism of section is described in this section. Each individual section starts with a section header which is followed by the data defined using the section type. Section headers are always aligned at $ 4-byte $ boundaries with respect to the start of the file image. In case of padding required between the section then to achieve the $ 4-byte $
alignment as defined, every bit value of padding is set to zero.
There some section types which are variable in terms of data length and are more precisely represented as data streams instead of data structures.

Irrespective of type of the section, all section headers starts with a $ 24-bit $ integer telling the section size, and $ 8-bit $ section type. The format of the rest of the section header and data is defined by the section type. If size of the section size is $ 0xFFFFFF $ then the size is defined by a $ 32-bit $ integer that follows the $ 32-bit $ section header. Figure \ref{fig:architecture-format-of-a-section-below-16Mb} and Figure \ref{fig:architecture-format-of-a-section-above-16Mb} shows the typical layout of a section data format.

\begin{figure}[!htbp]
	\centering
	\includegraphics[width=0.8\linewidth]{architecture/format-of-a-section-below-16Mb}
	\caption{Section Header Format when $ size < 16Mb $}\label{fig:architecture-format-of-a-section-below-16Mb}
\end{figure}

\begin{figure}[!htbp]
	\centering
	\includegraphics[width=0.8\linewidth]{architecture/format-of-a-section-above-16Mb}
	\caption{Section Header Format of when $ size \geq 16Mb $  using Extended Length field}\label{fig:architecture-format-of-a-section-above-16Mb}
\end{figure}


\subsection{File System Initialization}\label{subsection:file-system-initialization}
To ensure unity of the file system it is mandatory to maintain state byte of each file correctly such that it won't compromised even in case of power failure during operation on any FFS. It is desired that an FFS driver produces an instance of Firmware Volume Protocol so that every normal file operations carried out in that context. Every file operations has to follow all the rules of creation, update, and deletion mentioned in this specification to avoid corruption of the file system.

\subsection{Traversal and Access to Files}
The Security (SEC), PEI, and early DXE code needs to be capable to traverse the FFS such that it's read and execute operation on files carried out before a write-enabled DXE FFS driver is started it's execution so that the FFS may not have any inconsistencies because of any kind of previous system failure. Hence, it has to follow a set of rules to assert the credibility of files before using them. It is not incumbent on SEC, PEI, or the early read-only DXE FFS services to make any effort to perform recovery or modification the file system. If any case exists where execution can not continue because of inconsistencies in file system, a recovery boot must be initiated.

As there is one mutual exclusiveness that the SEC, PEI, and early DXE code can affect without instantiating a recovery boot. This condition can be summoned by any previous system failure such as power failure that come along while a file update on a previous boot. In such case, a failure can cause two files with an identical file name GUID to coexist within the same firmware volume where one of them will have the \verb|EFI_FILE_MARKED_FOR_UPDATE| bit set to its state field but are going to be otherwise totally valid file. The another file may be in unknown state of building up to and including \verb|EFI_FILE_DATA_VALID|. All files used preceding to the initialization of the write-enabled DXE FFS driver must be filtered with this test prior to their use. If this condition is observed, it's tolerable to trigger a recovery boot and allow the recovery DXE to perform the completion of update.

%\inputminted{c}{code/architecture-traversal-and-file-access.c}\label{code:architecture-traversal-and-file-access}


\paragraph{Note} There's no ascertain for redundant files once a file found in the \verb|EFI_FILE_DATA_VALID| state. The condition where two files in same firmware volume coexist having the same file name GUID and both are within the \verb|EFI_FILE_DATA_VALID| state cannot occur if the set of rules for creation and update are strictly followed.

\subsection{File Integrity and State}
File corruption, no matter the cause, must be detectable in order to carried out appropriate steps for file system repair. File corruption can come from various sources but broadly falls into three categories listed below:
\begin{itemize}
	\item Any general failure
	\item Failure on erase
	\item Failure on write
\end{itemize}

A general failure is characterized to be evidently random corruption of the storage media. This corruption can occur because of the design problem or obsolete storage media i.e. This type of failure can be as perceptive as replacing any single bit inside the file content. Using a good design of system along with reliable storage media, general failures can be avoided. However, the FFS enables catching of this kind of failure.

An erase failure happens when a block erase of firmware volume media isn't completed because of any system failure i.e. power failure. As the erase operation is not outlined, it is likely that most of the implementation of FFS that allow file write and delete operations will also develop a mechanism to rectify deleted files and unite free space. In case the operation is not carried out successfully, the file system can be left out in a state which is not consistent. 

Likewise, a write failure takes place when a file system write is in motion and is left incomplete because of any system failure i.e. power failure. This type of failure can lead the file system to be in an inconsistent state.

All of these failures can be traced while FFS initialization is in progress hence depending on the cause of the failure, many recovery schemes can be carried out.




\section{System Management Mode (\gls{smm})}\label{section-smm}
\subsection{Overview}
On \say{IA-32 processors}, System Management Mode (\gls{smm}) is a manner of operation which is different from flat modal so as from protected mode of the DXE and PEI phases. It is outlined as a real mode environs with $ 32-bit $ data bus access and its carried out in effect to either with a specified interrupt type or with the System Management Interrupt (\gls{smi}) pin. Note that Operation mode of SMM is OS independent mode and is discrete operational mode, however it may also lies in both within and OS runtime.

\begin{figure}[!htbp]
	\centering
	\includegraphics[width=0.9\linewidth]{smm/framework-smm-architecture}
	\caption{SMM Framework Architecture \cite{beyond-bios}}\label{fig:framework-smm-architecture}
\end{figure}


\subsection{System Management System Table (\gls{smst})}
System Management System Table (\gls{smst}) is core mechanism of SMM handler to pass information and enabling activity.

SMST table allows access to service based on the SMST which also known as SMM Services. Driver can only use SMM services during execution inside context of the SMM. \verb|EFI_SMM_BASE_PROTOCOL.GetSmstLocation()| service used to discover the address of SMST.

SMST is a set of potentiality exported for utilization by any driver that is loaded into \gls{smram}. It's similar to the EFI System Table except that by design it's fixed set of services and data and also doesn't acknowledge to the resiliency of an EFI protocol interface.

SMM infrastructure component of Framework provides SMST, which manages:
\begin{itemize}
	\item Dispatching drivers inside SMM
	\item Allocation of memory (\gls{smram})
	\item Switching the framework in and out of the applicable SMM of the processor
\end{itemize}

\subsection{SMM and Available Services}\label{subsection-smm-available-services}

\subsubsection{SMM Services}\label{subsection-smm-services}
As EFI runtime drivers have their constraints, similarly the model of SMM  framework will have them too. Especially, dispatch of drivers in SMM won't be capable of using any core protocol service. However SMST-based services, called SMM Services allows the drivers to be access using an SMM identical of the EFI System Table, but services of the core protocol won't grantees the availability in runtime. As an alternative, the complete mass of EFI Boot Services and EFI Runtime Services can be available while the driver loading or "constructor" phase. 

With utilizing the visibility of constructor, SMM driver is capable to leveraging rich set of EFI service to perform:
\begin{itemize}
	\item Marshall interface for EFI services.
	\item Observing EFI protocols which are populated by other SMM drivers while in constructor phase.
\end{itemize}

For drivers while not in SMM and during the initial load inside SMM, EFI protocol database becomes quite useful by utilizing design.

Available services which are SMST-based includes:
\begin{itemize}
	\item Negligible, blocking kind of the device Input/Output protocol
	\item Memory allocator from SMRAM
\end{itemize}
These services are exposed through the entries present in the System Management System Table (SMST).

\subsubsection{SMM Library}\label{subsection-smm-lib}
During constructor phase of SMM driver inside SMM, all additional service within SMM Library (SMLib) are uncovered like EFI protocols. i.e., An identical status code in SMM is merely an EFI protocol with interface referencing an SMM-based driver service. To avoid error or information of progress during runtime, other SMM drivers also locates this SMM based status code.

\subsection{SMM Drivers}\label{subsection-smm-drivers}
\subsubsection{Process to Load Drivers in SMM}
Process to load driver modal in SMM is merely a DXE SMM runtime driver having a DEPEX (dependency expression) having at least \verb|EFI_SMM_BASE_PROTOCOL|. This kind of dependency is essential as the DXE runtime driver which is planned for SMM will utilize the \verb|EFI_SMM_BASE_PROTOCOL| to load up itself again in SMM and re-execute its entry point. Also, other SMM-loaded protocols permitted to be stayed in the DEPEX of specified SMM DXE runtime driver. Principle of the DXE Dispatcher is to verifying if the GUIDs to be consumed by the protocols does exists in the database of protocol and capable to identifying if the driver can be loaded or not.

After formerly loaded in SMM the DXE SMM runtime driver becomes capable utilize only minor set of services. While in its constructor entry point, the driver can use EFI Boot Services as it executes within space of boot service and SMM.

In secondary entry point in SMM driver is capable to perform:
\begin{itemize}
	\item Registration of an interface - In the formal protocol database, naming the SMM occupant interfaces with future-loaded SMM drivers
	\item Registration with the SMM codebase - For a callback hook in effect to an SMI pin stimulation or an SMI based interrupt message from outside of SMM Code (i.e. a boot service, runtime agent)
\end{itemize}

After this \textbf{constructor} phase in SMM, the SMM driver needs not be rely on any other boot services as the mode of operation to carrying out execution can move away from these services. Many EFI Runtime Services could possess the majority of their execution shifted into SMM and viewable runtime portion simply becomes a proxy which merely utilizes the \verb|EFI_SMM_BASE_PROTOCOL| to perform callback in SMM to carry out services. By Possessing a proxy which allows for a modal of sharing code blocks of error handling, like services for flash access and also the EFI Runtime Services \verb|GetVariable()| or \verb|SetVariable()|.

\subsubsection{SMM Drivers for IA-32}
In SMM the \say{IA-32 runtime drivers} can not called as on the image the action by \verb|SetVirtualAddress()| is performed. Hence, code segment that requires to be accessible among SMM and EFI runtime needs to be migrated in SMM.

\subsubsection{"Itanium® Processor Family" SMM Drivers}
From \say{Platform Management Interrupt (\gls{pmi})} the runtime drivers for the \say{Itanium® processor family} are called as if each of them is kind of \say{Position Independent Code (PIC) runtime driver}.

\subsection{SMM Protocols}\label{subsection-smm-protocols}
System Architecture of SMM broke in to below two parts:
\begin{itemize}
	\item \say{SMM Base Protocol} - exposed by the processor. This protocol is liable to perform: 
		\begin{itemize}
			\item To initialize the state of processor
			\item Registration of the handlers
		\end{itemize}
	\item \say{SMM Access Protocol} - interprets the specific enabling and locking mechanism that an IA-32 memory controller may allows during execution in SMM. (Not needed for \say{Itanium® processor family})
\end{itemize}

\subsubsection{SMM Protocols for IA-32}
Figure \ref{fig:published-protocols-for-ia-32-systems} shows the SMM protocols which are published for an IA-32 system.
\begin{figure}[!htbp]
	\centering
	\includegraphics[width=0.9\linewidth]{smm/published-protocols-for-ia-32-systems}
	\caption{Protocols Published for IA-32 Systems}\label{fig:published-protocols-for-ia-32-systems}
\end{figure}

\subsubsection{SMM Protocols for "Itanium®-Based Systems"}
Figure \ref{fig:published-protocols-for-itanium-systems} shows the way SMM protocols are published for an \say{Itanium®-Based system}.
\begin{figure}[!htbp]
	\centering
	\includegraphics[width=0.9\linewidth]{smm/published-protocols-for-itanium-systems}
	\caption{Protocols Published for "Itanium®-Based Systems"}\label{fig:published-protocols-for-itanium-systems}
\end{figure}


\subsection{SMM Dispatcher and infrastructure}
SMM Code segment lies within the SMM Dispatcher. Major Role of SMM Dispatcher is to give the control mode to the SMM handlers in a systematic methodology. SMM Infrastructure Code aids to drive communication for SMM to SMM. SMM handlers are PE32+ images.

\subsection{Initializing SMM Phase}
The SMM driver for the Framework is fundamentally a enrollment transport mode to dispatch the drivers in outcome to the:
\begin{itemize}
	\item System Management Interrupts for \say{IA-32}
	\item Platform Management Interrupts (PMIs) for \say{Itanium® processor family}
\end{itemize}

\subsection{Relation of "System Management RAM (\gls{smram})" to conventional memory}
Figure \ref{fig:smram-relationship-to-main-memory} shows relationship between SMRAM and main memory in IA-32. Where SMRAM is isolated secure part inside the conventional main memory.

\begin{figure}[!htbp]
	\centering
	\includegraphics[width=0.9\linewidth]{smm/smram-relationship-to-main-memory}
	\caption{SMRAM kinship with conventional memory}\label{fig:smram-relationship-to-main-memory}
\end{figure}

\subsection{Execution Mode of SMM on Processor}\label{subsection-processor-execution-mode}
SMM is acceded asynchronously with the ongoing main flow of program. SMM was primitively developed to be clear to the OS and provide a power management facility more transparent.

Preboot agents are responsible to initiate alternate uses of SMM which are:
\begin{itemize}
	\item Applicable Workaround for \gls{soc} exaggeration
	\item Logging of error(s)
	\item Security for the platform
\end{itemize}

A \gls{smi} can be launched by energizing either the SMI logic pin via dedicated on the board or by utilizing the local APIC.

\say{Itanium® architecture} possess no independent separate mode for processor for the tractability of interruption however it does supports \say{Platform Management Interrupt (\gls{pmi})} which indeed is a maskable interruption. However, there is this another way to enter PMI using a interrupt message on local \say{Streamlined Advanced Programmable Interrupt Controller (\gls{sapic})}.

This architecture informs a techniques to load modules of needful code segment that substantiate the functionality specified above. The internal representation of protocol which enables the loading of images of various handler and runs in normal memory of boot-services. Only the handlers does to run in \gls{smram}.

\subsection{Accessing Platform Resources}\label{subsection-access-to-platform-resources}
As par policy outcome process of the execution of SMM handlers is reasonably prevents from accessing conventional memory resources. Hence, there does not exist any ease binding technique such as a call or trap interface to render the services in preemptive bid of non-SMM state.

Besides, SMM Services - the library of service which abides a sub set of the core EFI services, i.e. device input-output protocol, memory allocation and others. Also, execution mode of SMM driver has the equivalent structure as per the EFI criterion - namely a components which executes under boot services and it could perhaps run in runtime mode. When \verb|ExitBootServices()| invoked, the mechanism of an unregister event occurs.




\chapter{Implementation}\label{chapter-implementation}
\lhead{Chapter 3. \emph{Implementation}}

\section{Proposed Work}\label{section-proposed-work}
In general to generate BIOS image (*.rom file), compilation of XYZ.c (source code) has to be done, this compilation not only involves compilation of DXE driver, PEI driver, EFI Application but also includes pre-processing checks, compression of raw files which takes huge amount of time depending on the system configuration. Implementation of this project aids in reduction of this compilation time.

\subsection{Stake holders}\label{subsection-stack-holders}
The proposed work is applicable but not limited to below stake holders:
\begin{itemize}
	\item \textbf{BIOS development team} : main development group in contributing BIOS firmware, this is the only stake holder who are having access to the BIOS development environment and access to the source code of the complete BIOS firmware
	\item \textbf{Validation team} : performs various validation on developed BIOS image
	\item \textbf{Automation team} : brings various integration and validation automation to module(s)
	\item Other Development team who wishes to ease the debugging process
\end{itemize}

\subsection{Issues}\label{subsection-issues}
The Proposed work is capable of mitigating below issues:
\begin{itemize}
	\item Generation of BIOS image - includes compilation of whole source code
	\item Time complexity - took enormous amount of time to generate the BIOS image
	\item Accessing and modifying BIOS Setup Option(s) remotely
	\item Firmware Flashing of BIOS remotely
	\item Updating CPU microcode
	\item Summarizing changes among BIOS image
	\item Avoiding exposing the source code support for OEM to fill their OEM information
	\item Avoid setting of BIOS development platform for stake holders which are not meant to be the BIOS developer
	\item Runtime BIOS Support for temporary UEFI variable creation
\end{itemize}

\subsection{Requirements}\label{subsection-requirements}
\subsubsection{Software Requirements}
\begin{itemize}
	\item Visual C/C++ binaries
	\item Python 3
	\item Visual Studio Code (IDE)
	\item Memory Access Interface - supported mechanism to communicate over target memory
\end{itemize}

\subsection{Development Process of Modules}
Framework development process is driven by implementation of independent modules which can serve functionality and having flexibility to integration to the framework.


% ========================================================================
% MODULE 1
% ========================================================================

%\subsection{Module 1: Processing Firmware individually}\label{subsection-processing-firmware}
%To process apply the whole firmware changes individually for BIOS, signing is require to be performed for individual Firmware before stitching it to the BIOS binary in the valid structure.
%
%\subsubsection{Primary Goals of the module}
%\begin{itemize}
%	\item Remove other Intellectual Property's dependency (\gls{ip} dependency) during firmware loading
%	\item \gls{ip} Subsystem :
%	\begin{itemize}
%		\item Loader and Verifier
%		\item \gls{ip} is always consumer
%	\end{itemize}
%	\item Signature verification using \gls{sha} hash algorithm and should be ease support for adding new algorithmic support as needed.
%	\item Should support hardware based and software based verification support
%	modifying memory requirements for given IP without impacting eco-system
%	\item Prevent common security threats
%	\item Allow easier OEM adoption and modification based on the respective design
%	\item Reusability/Portability of design across many \gls{ip}s
%	\item Generic design which supports any new IP integration
%\end{itemize}
%
%\begin{figure}[!htbp]
%	\centering
%	\includegraphics[width=\linewidth]{proposed-work/proposed-structure-firmware-signing}
%	\caption{Proposed Structure for firmware signing}\label{fig:proposed-work-proposed-structure-firmware-signing}
%\end{figure}
%
%Note: Due to Confidentiality other details of this module is not to be disclosed such as flow diagram, structures or pseudo code.

% ========================================================================
% MODULE 1
% ========================================================================
\subsection{Module: Setup Knob modification}\label{module-setup-knob-modification}

\subsubsection{Processing Unsigned debug BIOS}\label{subsection-processing-bios}
Before Releasing the BIOS firmware for public use, those are signed for security and integrity purpose, however the debug BIOS which are used Pre-release to test and verify all the functional features until all the requirements are met.

Every \gls{soc} system which are under test known is \gls{sut} are configured in such a way that it supports debug BIOS. The proposed framework is designed to simulate the process of \gls{sut} in terms of processing BIOS binary similar to \gls{sut} performs it after flashing BIOS firmware on \gls{soc}.

Processing the debug BIOS can be classified in to two ways:
\begin{enumerate}\label{cli-classification-proposed-work}
	\item Applying changes directly to the \gls{sut}
	\item Applying changes on to the BIOS image
\end{enumerate}

At the high level the flow for both the above classification remains the same but will be differentiated at the backend support. An additional driver is attached with BIOS firmware to aid the framework to be able to apply changes directly to the \gls{sut}.

\subsubsection{Additional Tech Stack Used}
Below are the listed technologies consumed in development of this module in addition to the already specified requirements in Section \ref{subsection-requirements}
\begin{itemize}
	\item Tkinter
	\item XML
	\item JSON
\end{itemize}

\subsubsection{Flow of the module}
Figure \ref{fig:setup-knobs-flow} describes the flow of setup knobs modification on the System Under Test (SUT).

\begin{figure}[!htbp]
	\centering
	\includegraphics[width=0.9\linewidth]{proposed-work/setup-knobs-flow}
	\caption{Flow of Setup Knobs Modification}\label{fig:setup-knobs-flow}
\end{figure}


\subsubsection{Screenshots of Module}
As a \gls{poc} for the framework, this section shows snapshots of the working module to mimic the setup options of BIOS, however as a simulation framework, it also provides quite more features which are not available in the actual BIOS due to memory limitation.

Figure \ref{fig:proposed-work-bios-gui-initial-config} shows the prompt asked to user to select basic configurations before launching the module of framework. Configurations available to select are:

\begin{itemize}
	\item Working Mode (options to be selected as in figure \ref{fig:proposed-work-bios-gui-initial-config-select-mode})
	\begin{itemize}
		\item \verb|online| - to work on \gls{sut} and require to select valid access method for online mode from menu
		\item \verb|offline| - to work on BIOS binary
	\end{itemize}
	\item Access Method - selecting valid access method for working on \gls{sut}
	\item Publish all? - Boolean options to decide whether to evaluate \gls{depex} or not.
\end{itemize}

\begin{figure}[!htbp]
	\centering
	\includegraphics[width=0.7\linewidth]{proposed-work/bios-gui-initial-config}
	\caption{Menu to Select initial configuration for work}\label{fig:proposed-work-bios-gui-initial-config}
\end{figure}

\begin{figure}[!htbp]
	\centering
	\includegraphics[width=0.7\linewidth]{proposed-work/bios-gui-initial-config-select-mode}
	\caption{Available work mode for the system: Online and Offline}\label{fig:proposed-work-bios-gui-initial-config-select-mode}
\end{figure}


%Figure \ref{fig:proposed-work-bios-gui-acpi-knobs} shows the view that how an options in the module simulation is loaded.

\begin{table}
	\centering
	\renewcommand{\arraystretch}{2}
	\caption{Interpretation of buttons on Virtual Setup Page GUI}\label{table:interpretation-of-buttons-in-module}
	\begin{tabular}{l | p {8cm}}
		Button & Interpretation
		\\ \hline \hline
		Push Changes & Apply changes to system if online mode else apply changes to `bin` file
		\\ \hline View Changes & View saved changes in new window
		\\ \hline Exit & Exit the GUI
		\\ \hline Reload & Reload the GUI
		\\ \hline Discard Changes & Discard any change made, any value if modified are restored to current value
		\\ \hline Load Defaults & Restore to default values and revert any changes made
		\\ \hline
	\end{tabular}
\end{table}

%\begin{figure}[!htbp]
%	\centering
%	\includegraphics[width=0.8\linewidth]{proposed-work/bios-gui-acpi-knobs}
%	\caption{Setup Options listed under ACPI Configurations}\label{fig:proposed-work-bios-gui-acpi-knobs}
%\end{figure}

Table \ref{table:interpretation-of-buttons-in-module} describes the interpretation of each button action on specific condition as remarks if applicable


%Navigation through the various BIOS pages can be done as shown in Figure \ref{fig:proposed-work-bios-gui-accessing-menu}.

%\begin{figure}[!htbp]
%	\centering
%	\includegraphics[width=0.8\linewidth]{proposed-work/bios-gui-accessing-menu}
%	\caption{Navigating through BIOS setup page}\label{fig:proposed-work-bios-gui-accessing-menu}
%\end{figure}

%Figure \ref{fig:proposed-work-bios-gui-view-changes} displays list of changes made in setup options across any setup page and listed separately to compare with previous value, discard or apply the new values.

%\begin{figure}[!htbp]
%	\centering
%	\includegraphics[width=\linewidth]{proposed-work/bios-gui-view-changes}
%	\caption{Viewing all the changes made during current session}\label{fig:proposed-work-bios-gui-view-changes}
%\end{figure}

\subsubsection{Outcome of Module}
\begin{itemize}
	\item The module is capable of cross platform usage.
	\item The module can work with all the platform binary and \gls{sut}.
	\item A communication bridge as a driver in BIOS firmware to aid the framework run directly on \gls{sut} is implemented.
	\item Generic solution is provided for end-user while running any of the classification listed in \ref{cli-classification-proposed-work}.
	\item Simulating the information from system or binary image is provided as native GUI application.
	\item Real time sync with simulation framework is supported.
	\item Seamless Integration of any new features or modules in framework is made possible.
\end{itemize}

% ========================================================================
% MODULE 2
% ========================================================================
\subsection{Module: Parsing}\label{module-parsing}
Figure \ref{fig:bios-as-filesystem} represents the overview of the BIOS as a File system  which is interpreted and parsed from the BIOS image. Detail architecture of the same is explained in Section \ref{section-architecture}

\begin{figure}[!htbp]
	\centering
	\includegraphics[width=\linewidth]{proposed-work/bios-as-filesystem}
	\caption{Overview of BIOS image as a File System}\label{fig:bios-as-filesystem}
\end{figure}

\subsubsection{Additional Tech Stack Used}
Below are the listed technologies consumed in development of this module in addition to the already specified requirements in Section \ref{subsection-requirements}
\begin{itemize}
	\item Decompression binaries
	\item XML
	\item JSON
\end{itemize}

\subsubsection{Flow of the module}
Figure \ref{fig:uefi-parser} describes the flow of the Parsing module. The Initial part is performed by user who is responsible to select valid memory interface to work. Note that some memory interface are supported by the module which requires additional hardware and software setup which are considered to be the part of dependency of interface itself which is not in the scope of the module.

When User select valid Interface the module will determine whether user is on Target \gls{sut} or on the local BIOS image.
If user is working on SUT with valid memory interface and privileges then BIOS image will be parsed from the memory.

\begin{figure}[!htbp]
	\centering
	\includegraphics[width=\linewidth]{proposed-work/uefi-parser}
	\caption{Flow of Parser}\label{fig:uefi-parser}
\end{figure}

As on both the cases BIOS Image is available to act on, the module will start the parsing of the BIOS image as interpretation described in Figure \ref{fig:bios-as-filesystem}. It parses All the valid firmware volumes only till the end of BIOS image (skips the free space or firmware volumes with invalid signature and GUID). Decompression of file system under the firmware volume if any is handled by the module too, for the decompression of file system it uses the binary for decompression technique available to public i.e. lzma, tianocore, brotli etc.


\subsubsection{Outcome of Module}
\begin{itemize}
	\item Human Readable interpretation of BIOS image is provided.
	\item Possible to debug the BIOS via setup knobs comparison.
	\item Lookup of order of the module in BIOS image as readable file system is also possible.
	\item Verification of integration of module via GUID can be done.
	\item Extracting and storing file system or module of BIOS image by GUID
	\item Summarizing changes of two BIOS image
\end{itemize}



% ========================================================================
% MODULE 3
% ========================================================================
\subsection{Module: Runtime UEFI variable Creation}\label{module-runtime-uefi-variable-creation}
Each variable in BIOS has a scope for each variable where Runtime support is one of the attribute, to simply state the run time variable one can interpret it as the variable which will be available during and after the completion boot flow (while OS is running). Such a variable require special access mechanism, which is carried out by the System Management mode \gls{smm} described in Section \ref{section-smm}.

Earlier Challenges are described as below:
\begin{itemize}
	\item Providing and maintaining native driver support from BIOS for creation of UEFI variable
	\item Setting of Build environment for non-BIOS development team
\end{itemize}

Note: As all the variable created at runtime the scope of such variable are limited to the flashing of the BIOS. i.e. when BIOS is flashed/re-flashed or updated, those variable won't be available on the SUT.

\subsubsection{Additional Tech Stack Used}
Below are the listed technologies consumed in development of this module in addition to the already specified requirements in Section \ref{subsection-requirements}
\begin{itemize}
	\item Flask
	\item Ajax
	\item jQuery
	\item Javascript
	\item HTML/CSS
	\item XML
	\item JSON
\end{itemize}

\subsubsection{Flow of the module}
\begin{figure}[!htbp]
	\centering
	\includegraphics[width=1\linewidth]{proposed-work/nvar_web_GUI_flow}
	\caption{Flow of Nvar Web GUI}\label{fig:nvar_web_GUI_flow}
\end{figure}
The Flow of the module is described in section \ref{subsubsection-screenshots} along with screenshots which is easier to interpret the flow chart in Figure \ref{fig:nvar_web_GUI_flow}.

\subsubsection{Screenshots of Module}\label{subsubsection-screenshots}
Whenever the User launches the module the home page screen to select valid communication interface will appear as displayed in Figure \ref{fig:uefi-variable-home}. This is the crucial stage as if valid interface for communication is not selected one may not be able to use the functionality of the service.

\begin{figure}[!htbp]
	\centering
	\includegraphics[width=\linewidth]{proposed-work/uefi-variables/home}
	\caption{Home Page to Create UEFI Variable}\label{fig:uefi-variable-home}
\end{figure}

After selection of valid Interface one may operate the desired options listed in navigation bar which are:

\begin{table}
	\centering
	\renewcommand{\arraystretch}{2}
	\caption{Navigation Bar Action}\label{table:navbar-action}
	\begin{tabular}{l | p {8cm}}
		Button & Interpretation
		\\ \hline \hline
		Create Variable & Opens a form to create new Variable as in Figure \ref{fig:uefi-variable-create-nvar}
		\\ \hline Display Created Variable & lists out created variable as in Figure \ref{fig:uefi-variable-created-option}
		\\ \hline Generate XML & Generate XML from the stored session database as in Figure \ref{fig:uefi-variable-generate-xml}
		\\ \hline Save XML & Saves the generated XML on the storage device
		\\ \hline Save to SUT & Applies the Pending changes action (Create/Delete/Modify) to the \gls{sut}
		\\ \hline View JSON & View the stored session database in the json format as in Figure \ref{fig:uefi-variable-represent-json}
		\\ \hline
	\end{tabular}
\end{table}


Figure \ref{fig:uefi-variable-created-option} lists the variable created under the current session which is to be applied 
\begin{figure}[!htbp]
  \centering
  \includegraphics[width=\linewidth]{proposed-work/uefi-variables/created-option}
  \caption{Variables created or exists on \gls{sut}}\label{fig:uefi-variable-created-option}
\end{figure}

Figure \ref{fig:uefi-variable-create-nvar} displays form which allows user to create Variable, where user needs to specify the name of the variable with certain restriction of input field. To identify and lookup the Variable the GUID is required which is automatically generated by the module with required format, however if user wishes then they can modify the GUID.
\begin{figure}[!htbp]
  \centering
  \includegraphics[width=\linewidth]{proposed-work/uefi-variables/create-nvar}
  \caption{Create new UEFI Variable on \gls{sut}}\label{fig:uefi-variable-create-nvar}
\end{figure}


Figure \ref{fig:uefi-variable-var-options} opens the list of the options if created and allows to edit their current values too. However one can also add the new option to the Variable. It allows user to create various types of options under the variable which are oneof type as in Figure \ref{fig:uefi-variable-add-option}, string type as in Figure \ref{fig:uefi-variable-string}, numeric type as in Figure \ref{fig:uefi-variable-numeric} and the checkbox type which allows user to toggle the option value in as Boolean interpretation. Common fields for creating options including its name, type, description and size.
\begin{figure}[!htbp]
  \centering
\end{figure}

If user wants to change the value set for the variable created while creation of option as described in Figure \ref{fig:uefi-variable-edit-option} forms one can actually modify the value.
\begin{figure}[!htbp]
  \centering
  \includegraphics[width=\linewidth]{proposed-work/uefi-variables/edit-option}
  \caption{Edit the Existing Option Created under Variable \gls{sut}}\label{fig:uefi-variable-edit-option}
\end{figure}

The highlighted prompt in Figure \ref{fig:uefi-variable-add-option} allows user to create the choices for the option where one of the multiple values to be selected as a result, By clicking \verb|Add Option| button user can create choices and under drop down menu besides Value, user can select default value to be selected for the option. 
\begin{figure}[!htbp]
  \centering
  \includegraphics[width=\linewidth]{proposed-work/uefi-variables/add-option}
  \caption{Create New Option(s) under Variable - Oneof Type}\label{fig:uefi-variable-add-option}
\end{figure}

Option type string as in Figure \ref{fig:uefi-variable-string} allows user to create a option which accepts minimum and maximum characters to be supported in the string as well as the default string value to be selected. 
\begin{figure}[!htbp]
	\centering
	\includegraphics[width=\linewidth]{proposed-work/uefi-variables/string}
	\caption{Create New Option(s) under Variable - String Type}\label{fig:uefi-variable-string}
\end{figure}

To set the numeric input for the option, minimum and maximum value along with the default value to be set as in Figure \ref{fig:uefi-variable-numeric}
\begin{figure}[!htbp]
  \centering
  \includegraphics[width=\linewidth]{proposed-work/uefi-variables/numeric}
  \caption{Create New Option(s) under Variable - Numeric Type}\label{fig:uefi-variable-numeric}
\end{figure}

For the future use one can create a reserved space under the UEFI variable as in Figure \ref{fig:uefi-variable-reserved-space}
\begin{figure}[!htbp]
  \centering
  \includegraphics[width=\linewidth]{proposed-work/uefi-variables/reserved-space}
  \caption{Create Reserved Space for future use under Variable}\label{fig:uefi-variable-reserved-space}
\end{figure}


Figure \ref{fig:uefi-variable-generate-xml} shows the XML which is generated from the existing and newly created variables and options under it.
\begin{figure}[!htbp]
  \centering
  \includegraphics[width=\linewidth]{proposed-work/uefi-variables/generate-xml}
  \caption{Generate XML \gls{sut}}\label{fig:uefi-variable-generate-xml}
\end{figure}

Figure \ref{fig:uefi-variable-represent-json} represents session data of existing and newly created data (if any) as json
\begin{figure}[!htbp]
  \centering
  \includegraphics[width=\linewidth]{proposed-work/uefi-variables/represent-json}
  \caption{View JSON representation \gls{sut}}\label{fig:uefi-variable-represent-json}
\end{figure}


\subsubsection{Outcome of the module}
\begin{itemize}
  \item Enables creation of UEFI variable from OS layer.
  \item Lifts headache of maintaining variable creation from BIOS development for individuals.
  \item Interactive Web GUI for better User Experience
  \item Quick Modification of already created Variable
\end{itemize}


\section{Future Scope of Work}\label{section-future-scope}
Few implementation modules of Section \ref{section-proposed-work} are not well developed at production launch which a slight modification and standard checks have to be performed to make the modules qualify for production level. Also as the release of production other stuff to be maintained is user guide, FAQs and other "how-to" articles to help out others to ease in using the framework.

Along with the enhancing of existing modules there will still be exercise to analyze existing system to explore more use cases which are taking a longer time for every build iteration for the system.

Few of the possible use cases to study and decide the feasibility of implementation would be:

\begin{itemize}
	\item Development and testing of individual driver component rather than building the whole BIOS image
	\item AI powered Search Engine to enhance the findings of FAQs for relevant existing queries and articles
	\item Automating the initial BIOS Environment Setup
	\item Platform independent easy installation setup for the framework
\end{itemize}



\clearpage
\printglossary[type=main]
%\printnoidxglossaries[type=main]
\end{document}